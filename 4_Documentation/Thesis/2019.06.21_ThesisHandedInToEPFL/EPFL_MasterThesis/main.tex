%%%% Time-stamp: <2018-03-24 14:05:09 vk>
%% ========================================================================
%%%% Disclaimer
%% ========================================================================
%%
%% created by
%%
%%      Karl Voit
%%

%% ========================================================================
%%%% Basic settings
%% ========================================================================
%% (idea of using newcommands for basic documentclass settings from: Thomas Schlager)

\newcommand{\mypapersize}{A4}
%% e.g., "A4", "letter", "legal", "executive", ...
%% The size of the paper of the resulting PDF file.

\newcommand{\mylaterality}{oneside}
%% "oneside" or "twoside"
%% Either you are creating a document which is printed on both, left pages
%% and right pages (twoside) or you create a document which is printed
%% on right pages only (oneside).

\newcommand{\mydraft}{false}
%% "true" or "false"
%% Use draft mode? If true, included graphics are replaced by empty
%% rectangles (of same size) and overfull boxes (in margin space) are
%% marked with black box (-> easy to spot!)

\newcommand{\myparskip}{half}
%% e.g., "no", "full", "half", ...
%% How to separate paragraphs: indention ("no") or spacing ("half",
%% "full", ...).

\newcommand{\myBCOR}{0mm}
%% Inner binding correction. This value depends on the method which is
%% being used to bind your printed result. Some techniques do not
%% require a binding correction at all ("0mm"), other require for
%% example "5mm". Refer to KOMA script documentation for a detailed
%% explanation what a binding correction is and how to measure it.

\newcommand{\myfontsize}{12pt}
%% e.g., 10pt, 11pt, 12pt
%% The font size of the main text in pt (points).

\newcommand{\mylinespread}{1.0}
%% e.g., 1.0, 1.5, 2.0
%% Line spacing in %/100. For example 1.5 means 150% of the usual line
%% spacing. Please use with caution: 100% ("1.0") is fine because the
%% font was designed for it.

\newcommand{\mylanguage}{ngerman,british}
%% "english,ngerman", "ngerman,english", ...
%% NOTE: The *last* language is the active one!
%% See babel documentation for further details.

%% BibLaTeX-settings: (see biblatex reference for further description)
\newcommand{\mybiblatexstyle}{authoryear}
%% e.g., "alphabetic", "authoryear", ...
%% The biblatex style which is being used for referencing. See
%% biblatex documentation for further details and more values.
%%
%% CAUTION: if you change the style, please check for (in)compatible
%%          "biblatex" package options in the file
%%          "template/preamble.tex"! For example: "alphabetic" does
%%          not have an option "dashed=..." and causes an error if it
%%          does not get removed from the list of options.

\newcommand{\mybiblatexdashed}{false}  %% "true" or "false"
%% If true: replace recurring reference authors with a dash.

\newcommand{\mybiblatexbackref}{true}  %% "true" or "false"
%% If true: create backward links from reference to citations.

\newcommand{\mybiblatexfile}{EPFL_MasterThesis.bib}
%% Name of the biblatex file that holds the references.

\newcommand{\mydispositioncolor}{30,103,182}
%% e.g., "30,103,182" (blue/turquois), "0,0,0" (black), ...
%% Color of the headings and so forth in RGB (red,green,blue) values.
%% NOTE: if you are using "0,0,0" for black, printers might still
%%       recognize pages as color pages. In case this is a problem
%%       (paying for color print-outs vs. paying for b/w-printouts)
%%       please edit file "template/preamble.tex" and change
%%       "\definecolor{DispositionColor}{RGB}{\mydispositioncolor}"
%%       to "\definecolor{DispositionColor}{gray}{0}" and thus
%%       overwriting the value of \mydispositioncolor above.

\newcommand{\mycolorlinks}{true}  %% "true" or "false"
%% Enables or disables colored links (hyperref package).

\newcommand{\mytitlepage}{template/title_Thesis_EPFL}
%% Your own or one of following pre-defined title pages:
%% "template/title_plain_maketitle": simple maketitle page
%% "template/title_Diplomarbeit_KF_Uni_Graz.tex": fancy (german) title page for KF Uni Graz
%% "template/title_Thesis_TU_Graz":
%%             titlepage for Graz University of Technology (correct
%%             (old?) Corporate Design) by Karl Voit (2012)
%% "template/title_Thesis_TU_Graz_-_kazemakase":
%%             titlepage for Graz University of Technology
%%             (correct new Corporate Design) by kazemakase (2013):
%%             see https://github.com/novoid/LaTeX-KOMA-template/issues/5
%% "template/title_VWA": titlepage for Vorwissenschaftliche Arbeit

\newcommand{\mytodonotesoptions}{}
%% e.g., "" (empty), "disable", ...
%% Options for the todonotes-package. If "disable", all todonotes will
%% be hidden (including listoftodos).

%% Load main settings for document preamble:
\input{template/preamble}%% DO NOT REMOVE THIS LINE!

\setboolean{myaddcolophon}{true}  %% "true" or "false"
%% If set to "true": a colophon (with notes about this document
%% template, LaTeX, ...) is added after the title page.
%% Please do not set to "false" without a good reason. The colophon
%% helps your readers to get in touch with LaTeX and to find this template.

\setboolean{myaddlistoftodos}{false}  %% "true" or "false"
%% If set to "true": the current list of open todos is added after the
%% table of contents. If \mytodonotesoptions is set to "disable", no
%% list of todos is added, independent of this setting here.

\setboolean{english_affidavit}{true}  %% "true" or "false"
%% If set to "true": the language of the statutory declaration text is set to
%% English, otherwise it is in German.


%% ========================================================================
%%%% Document metadata
%% ========================================================================

%% general metadata:
\newcommand{\myauthor}{Servan Grüninger}  %% also used for PDF metadata (hyperref)
\newcommand{\myauthorwithexistingtitles}{\myauthor{}, MSc Biostatistics}  %% including
                                %% university degree already held
                                %% (BSc, MSc, ...)
\newcommand{\mytitle}{Weight of Statistical Evidence}  %% also used for PDF metadata (hyperref)
\newcommand{\mysubtitle}{Detection and Correction of Publication Bias}  %% only used with title_Thesis_TU_Graz_-_kazemakase
\newcommand{\mysubject}{SUBJECT}  %% also used for PDF metadata (hyperref)
\newcommand{\mykeywords}{KEYWORDS}  %% also used for PDF metadata (hyperref)

%% this information is used only for generating the title page:
\newcommand{\myworktitle}{Master Project (2019), Mathematics Section}  %% official type of work like ``Master theses''
\newcommand{\mygrade}{Master of Science in \mystudy} %% title you are getting with this work like ``Master of ...''
\newcommand{\mystudy}{Computational Science and Engineering} %% your study like ``Arts''
\newcommand{\mydegreeprogramme}{Master's degree programme: \mystudy} %% Master's or PhD degree programme
\newcommand{\myuniversity}{École polytechnique fédérale de Lausanne} %% your university/school
\newcommand{\myfaculty}{ }  %% only used with title_Thesis_TU_Graz_-_kazemakase
\newcommand{\myinstitute}{Chair of Applied Statistics, EPFL} %% affiliation
\newcommand{\myinstitutehead}{Prof. Dr. Stephan Morgenthaler} %% head of institute
\newcommand{\mysupervisor}{Prof. Dr. Stephan Morgenthaler} %% your supervisor
\newcommand{\mycosupervisor}{}  %% only used with title_Thesis_TU_Graz_-_kazemakase
\newcommand{\myevaluator}{Prof. Dr. Stephan Morgenthaler} %% your evaluator
\newcommand{\myhomestreet}{Aarbergstrasse 95} %% your home street (with house number)
\newcommand{\myhometown}{Écublens} %% your home town
\newcommand{\myhomepostalnumber}{2502} %% your postal number of home town
\newcommand{\mysubmissionmonth}{June 21} %% month you are handing in
\newcommand{\mysubmissionyear}{2019} %% year you are handing in
\newcommand{\mysubmissiontown}{\myhometown} %% town of handing in (or \myhometown)


%% additional information for generic_documentation title page
\newcommand{\myid}{1234567} %% Matrikelnummer
\newcommand{\mylecture}{LECTURE} %%


%% ========================================================================
%%%% MISC command definitions
%% ========================================================================
\input{template/mycommands}

%% ========================================================================
%%%% Typographic settings
%% ========================================================================
\input{template/typographic_settings}


%% ========================================================================
%%%% MISC usepackages
%% ========================================================================
\usepackage{multirow,array}
\usepackage{amsmath}
\usepackage{amsthm}
\usepackage{epigraph}
\usepackage{listings} % to print code snippets
\setlength{\epigraphrule}{0pt}
\setlength{\epigraphwidth}{0.8\textwidth}
%\usepackage{caption} %% messes with "listoffigures" -not sure why though


%% ========================================================================
%%%% MISC self-defined commands and settings
%% ========================================================================
\DeclareMathOperator{\E}{\text{E}}
\DeclareMathOperator{\Var}{\text{Var}}
\DeclareMathOperator{\ppr}{\text{ppr}}
\newcommand{\ssqrt}[2][]{\sqrt[#1]{#2}\,}
\newcommand{\blb}{\left(}
\newcommand{\brb}{\right)}

%% Define a few struts
%% (from code by Claudio Beccari in TeX and TUG News, Vol. 2, 1993)
%% from https://tex.stackexchange.com/questions/287738/increase-the-height-of-the-row-in-the-table
\newcommand\Tstrut{\rule{0pt}{2.9ex}}       % "top" strut
\newcommand\Bstrut{\rule[-1.3ex]{0pt}{0pt}} % "bottom" strut
\newcommand\TBstrut{\Tstrut\Bstrut}         % "top and bottom" strut

%% ... it's OK to put here your own newcommand/newenvironment-definitions ...




%\newcommand{\myLaT}{\LaTeX{}@TUG\xspace} %% LaTeX@TUG text "logo"

\hyphenation{ex-am-ple hy-phen-ate}  %% in order to use German umlauts
%% here (Ver-\"of-fent-li-chung), you have to check for
%% activated \usepackage[T1]{fontenc} in the preamble

%% override default language of babel: (be sure to know, what you're
%% doing here)
%\selectlanguage{american}
%\selectlanguage{ngerman}

%% ========================================================================
%%%% Templates
%% ========================================================================

%% template for inserting figures:
% \myfig{}%% filename
%       {}%% width/height
%       {}%% caption
%       {}%% optional (short) caption for list of figures
%       {fig:}%% label

%% acronyms in small caps: \myacro{UNESCO}


\input{template/pdf_settings}  %% should be *last* definitions in preamble!
%% ========================================================================
%%%% begin{document}
%% ========================================================================
\begin{document}

\frontmatter                    %% KOMA: roman page numbers and such; only available in scrbook

%%%% Time-stamp: <2013-03-18 14:35:00 vk>
%% ========================================================================
%%%% Disclaimer
%% ========================================================================
%%
%% created by
%%
%%      Karl Voit
%%
%% adapted by
%%      Servan Grüninger

\newcommand{\mycolophon}{%%
  This document was written with \href{https://www.overleaf.com/}{Overleaf} and 
  is set in Palatino, compiled with
  \href{https://ctan.org/pkg/pdftex}{pdf\LaTeX} and
  \href{https://ctan.org/pkg/biber}{\texttt{Biber}}.
  
  It is an adapted version of the  \href{http://www.komascript.de/}{KOMA-script-based} template created by \href{https://karl-voit.at/}{Karl Voit}. The template can be found online on \href{https://github.com/novoid/LaTeX-KOMA-template}{Github}.
}

%%% Local Variables: 
%%% mode: latex
%%% mode: auto-fill
%%% mode: flyspell
%%% eval: (ispell-change-dictionary "en_US")
%%% TeX-master: "../main"
%%% End: 
                %% defines information about editor, LaTeX, font, ...

%% Choose your desired title page:
\input{\mytitlepage}            %% include title page

%% include the abstract without chapter number but include it on table of contents:
\cleardoublepage
\phantomsection
\addcontentsline{toc}{chapter}{Abstract}
%%%% Time-stamp: <2013-02-25 10:31:01 vk>


\chapter*{Abstract}
\label{cha:abstract}

Assessing the statistical evidence of scientific findings is challenging. Firstly, the construction of robust evidence measures can  be challenging and often hinges on a range of theoretical assumptions that might not be fulfilled in practice. Secondly, there are many different procedures to construct evidence measures which makes comparisons across studies difficult. Thirdly, the landscape of publication in science is heavily distorted by non-scientific incentives which gives rise to so-called `publication bias' and thus makes aggregation of evidence across studies even more challenging.\par 
In the first part of this thesis, I describe different methods to construct robust statistical evidence measures based on the idea of variance stabilising transformations. I then use these estimators in the second part to analyse and improve various methods for the detection and correction of publication. Theoretical arguments in combination which results from simulations show that the construction of robust evidence measures remains challenging and how statistical methods for the detection and correction of biases in scientific findings can be further improved.

%\glsresetall %% all glossary entries should be used in long form (again)
%% vim:foldmethod=expr
%% vim:fde=getline(v\:lnum)=~'^%%%%\ .\\+'?'>1'\:'='
%%% Local Variables:
%%% mode: latex
%%% mode: auto-fill
%%% mode: flyspell
%%% eval: (ispell-change-dictionary "en_US")
%%% TeX-master: "main"
%%% End:
              %% Abstract
\addcontentsline{toc}{chapter}{Acknowledgements}
\chapter*{Acknowledgements}
\label{cha:acknowledgements}
I thank my supervisor, Prof. Dr. Stephan Morgenthaler, for his guidance and feedback in the course of this thesis as well as the intellectual freedom I enjoyed. In addition, I thank the Swiss Study Foundation, and the Werner Siemens Foundation for their financial and non-material support. It has empowered me and given me the freedom to conduct my studies the way I wanted. Finally, I want to thank my fiancée for the many enthralling discussions about the philosophical aspects of scientific evidence.  
%\addcontentsline{toc}{chapter}{Author's note}
%\chapter*{Author's note}
\label{cha:author's note}

Redo: 
- chapter about hypothesis test, especially \ref{subsec:hypothesis_types}

All chapters contain a quote fitting its topic. I tried to be as specific and accurate as possible regarding the sources of the quotes. Whenever possible, I not only added the author, but also the year as well as the specific publication the quote can be found in. In some cases, this was not possible, so I had to heed the advice of the following quote:
\begin{figure}[htp]
  \centering
  \includegraphics[width=.8\textwidth]{figures/if-its-on-the-internet.jpg}
  {\caption{}}
  \label{fig:if it's on the internet} %% NOTE: always label *after* caption!
\end{figure}

Random variables are represented by uppercase letters, realisations of random variables are denoted by lowercase letters.
Exceptions: when the random variable is a function that was introduced with lowercase letters, lower case letters are used (such as $p_pub$ > but need to assess whether this is correct
also, variables and functions are written in italicis except for variables and function which are abbrevieations, they are written in standard font

further exception: $pp$-values
exception $N$ denotes total of all $n$

Notation: 
Expextactations are written as E[$\cdot$]
Probabilities are written in italics P($\cdot$)

todo: 
- check all "nodate" (n.d.) bibliographies and check consistency of bibliograph
- use clear uppercase \& lowercase rule for sampling stattistics and make sure you use sampling distribution when it is needed: %https://en.wikipedia.org/wiki/Sampling_distribution
Also, use standard error where neede
check textup vs test: %https://tex.stackexchange.com/questions/33120/should-subscripts-in-math-mode-be-upright
$S_n$ denotes a random summary or test statistic
$Z_n$ denotes a the z-statistic
$T_n$ denotes the t-statistic
$V_n$ denotes an evidence measure that is a variance stabilised transformation of a test statistic $S_n$ fulfilling desirable propertie $E_1$ to $E_4$. a random variance stabilised test statistic based on estimated population variance

For weird formatting, see here: %https://tex.stackexchange.com/questions/736/pagebreak-vs-newpage            %% Author's note
% \input{thanks}                %% this is a suggestion: you have to create this file on demand
% \input{foreword}              %% this is a suggestion: you have to create this file on demand

\tableofcontents                %% this produces the table of contents - you might have guessed :-)

\listoffigures
\listoftables

%% if myaddlistoftodos is set to "true", the current list of open todos is added:
\ifthenelse{\boolean{myaddlistoftodos}}{
  \newpage\listoftodos          %% handy if you are using todonotes with \todo{}
}{}                             %% with todonotes-package option "disable" you can get rid of any todo in the output

\mainmatter                     %% KOMA: marks main part using arabic page numbers and such; only available in scrbook

%\input{0_MetaAnalysis}   %% remove this line to get rid of the MetaAnalysis chapter
%\input{1_SimultaneousStatisticalInference}
%\input{2_CombiningEvidence}
\chapter{Introduction}
%\epigraph{\centering \textit{`Averages and relationships and trends and graphs are not always what they seem. There may be more in them than meets the eye, and there may be a good deal less.'}}{--- Darrell Huff,\\ How to Lie with Statistics (1954, p.~8)}
%\epigraph{\centering \textit{`[W]hen it comes to making decisions and statistical inferences, if you can't count you don't count.'}}{--- Stephen Senn, Dicing with Death (2003, p.~xi)}
\epigraph{\centering \textit{`I've been studying statistics for over 40 years \& I still don't understand it. The ease with which non-statisticians master it is staggering.'}}{--- Stephen Senn, Twitter, (\href{https://twitter.com/stephensenn/status/538017638111531009}{November 27, 2014})}

Complaints about statistics (and statisticians) are legion. In the best case, scientists simply find it boring, tedious and obfuscated, but recognise its worth in uncovering scientific facts. In the worst case, they see it as a mere means to an end, the end often being significant findings or findings going into the `desired direction'. Everyone wants evidence, but nobody wants to work for it---or at least collaborate with the ones who are willing to work the stats.\par
This might seem a bit caustic but it exemplifies the reactions that statistics usually evokes. And it it should make clear that many of the problems of scientific research---including the eponymous bias of this thesis---have less to do with statistics and are more deeply rooted in distorting external and internal incentives. Hence, it would be short-sighted to believe that statistical methods will be able to solve problems in research practice without major changes in research culture and specifically the current reward structures within scientific research.\par
Nevertheless, it is crucial to provide a solid statistical basis for the analysis and interpretation of scientific evidence in the face of uncertainty. Therefore, Chapter~\ref{cha:analysing_testing_evidence} provides an overview of basic methods to test hypotheses and compare statistical evidence from different sources, primarily focusing on the idea of variance stabilising transformations and their advantageous properties for testing and comparing scientific findings.\par
Whereas these methods primarily focus on how to calculate comparable evidence measures from a single study, the remainder of the second part of the thesis is concerned with how to combine statistical evidence from multiple studies in the presence of bias. This is crucial to draw the correct inferences about statistical parameters, which is, as William Sealy Gosset puts it in his seminal paper `The probable error of a mean' \citep[p.~1]{student_probable_1908}, the main purpose of scientific experiments: `Any series of experiments is only of value in so far as it enables us to form a judgement as to the statistical constants of the population to which the experiments belong'.\par
It follows then that incomplete and biased publication records heavily undermine the value of scientific research because they prevent a reliable `judgement as to the statistical constants of the population'. Therefore, Chapter~\ref{cha:publication bias} introduces the reader to a variety of sources behind publication bias in general and significance-driven publication bias in particular (Section~\ref{sec:significance_fickle}).\par
In addition, I present a selection of methods to detect (see Section~\ref{sec:detect_publication_bias}) and correct (see Section~\ref{sec:correct_publication_bias}) publication bias.\par
Finally, Chapter~\ref{cha:conclusion_and_limitations} gives the reader a quick overview of additional steps and methods to be used for the detection and correction of publication bias.
\chapter{Analysing and Testing Evidence}
\label{cha:analysing_testing_evidence}
\epigraph{\centering \textit{`Use the CRS database to size the market.' -- `That data is wrong.'}\par
\textit{`Then use the SIBS database.' -- `That data is also wrong.'}\par
\textit{`Can you average them?' -- `Sure. I can multiply them too.'}}{--- Dilbert by Scott Adams \href{https://dilbert.com/strip/2008-05-07}{May 07, 2008}}

Good statistics is like strong coffee: bitter at times but always sobering. And there are few things more sobering than taking a cherished hypothesis and putting it to the test---especially, if the test turns out to be negative. If used correctly, rigorous statistics is the ultimate tool for the `skillful interrogation of Nature', as \citet[p.~140]{fisherbox_fisher_1978} puts it: It helps to uncover relationships and connections which do not immediately catch the eye and prevents over-excited researchers from clinging to spurious findings.\par
Clever designs and nimble mathematical manipulations can make complex problems much more accessible and prevent common pitfalls. The next few sections are dedicated to a short overview of the basics of some of the many methods used for analysing and testing statistical evidence stemming from scientific data.

\section{Hypothesis testing}
Researchers often have to address questions such as the following:
\begin{align*}
    \text{Is treatment A superior to treatment B?} 
\end{align*}
In order to address this question, one usually formulates two distinct hypotheses:
\begin{align*}
    H_0 &: \text{Treatment A is not superior to treatment B}.\\
    H_1 &: \text{Treatment A is superior to treatment B}.
\end{align*}
To translate these hypotheses into a mathematically feasible language, one needs to operationalise them by establishing the criteria for `superiority' of a treatment. This can be done by defining primary (and potentially secondary) outcomes that measure the treatment effect of treatments A and B. What these primary outcomes are should be decided based on the specific question at hand and the expert knowledge of the involved researchers.\par
In other words, one must define a parameter $\theta$ with which one can quantify `superiority' and that allows us to reformulate the hypothesis as follows:
\begin{align*}
    H_0 &: \theta_A \leq \theta_B\\
    H_1 &: \theta_A > \theta_B
\end{align*}
%toadd: maybe add a chapter about Neyman-Pearson?
For illustrative purposes, let us assume that we are dealing with two different cancer treatments. One primary outcome could be the proportion of patients still alive after a given time period. This example is assessed in Section \ref{sec:binomial_var}. Another primary outcome could be the average survival time of the patients after receiving the treatment. This example is further commented on in Section \ref{sec:mean}.\par

\subsection{Types of hypotheses and hypothesis tests}
\label{subsec:hypothesis_types}
Before I can address the question of how to test these hypotheses, I first need to make a small detour to delve deeper into the specific properties a hypothesis can have. Above---and for the remainder of this Master thesis---I let the hypotheses in question be `composite'. However, hypotheses can also be `simple', with the distinction lying in the number of possible values for $\theta$ that a hypothesis specifies.\par
A hypothesis is called simple if it uniquely identifies the parameter $\theta$ (or the probability distribution specified by $\theta$). Conversely, if a hypothesis states more than one possible parameter value, it is called composite because it is composed of multiple simple hypotheses.\par
For illustration, let us consider a set of hypotheses that is different from the one stated above:
\begin{align*}
    H_0 &: \theta = \theta_0 = 0 \\
    H_1 &: \theta > \theta_0 = 0
\end{align*}
In this case, the null hypothesis $H_0$ is simple: For it to be true, $\theta$ can only correspond to a single value. Conversely, the alternative hypothesis $H_1$ is composite because it specifies more than one value for $\theta$: $H_1$ is true for any ${\theta > 0}$, regardless of whether this corresponds to ${\theta = 10^{-3}}$ or ${\theta = 10^3}$.\par
Now, in order to test the hypotheses, be they simple or composite, there are, in principle, three options:
\begin{itemize}
    \item Perform a one-sided test for superiority: Is $\theta$ larger than $\theta_0$?
    \item Perform a one-sided test for inferiority: Is $\theta$ smaller than $\theta_0$?
    \item Perform a two-sided test: Is $\theta$ either smaller or larger than $\theta_0$?
\end{itemize}
%toadd: chapter about difference between Fisher and Neyman-Pearson framework > uniformly most powerful test concepts, problems for one and two-sided hypotheses
These tests can be performed for three different scenarios: for two simple, a simple and a composite or two composite hypotheses. For the remainder of this Master thesis, I will let both $H_0$ and $H_1$ be composite hypotheses and explain all methods using the example of a one-sided test for superiority.\par

\subsection{Test statistics and decision functions}
\label{subsec:decision_function}
To test a specific set of hypotheses, one needs to construct a test statistic $S_n(\cdot)$, that is, a function that takes as input random variable $X_1 \dots X_n$ and provides as output another random variable whose realisation serves to distinguish between $H_0$ and $H_1$ given a decision function $\delta(\cdot)$. Note that the test statistic can but does not have to be an estimator of $\theta$. For example, the exact binomial test described in Section~\ref{subsec:exact binomial test} is based on a test statistic that is at the same time an estimator of $\theta$. In many cases, however, a test statistic consists of a transformation of the estimator, see for example the $Z$-statistic in Section~\ref{subsec:z-score} or the Student's $t$-statistic in Section~\ref{subsec:student's t-statistic}.\par
To perform a one-sided test for superiority on the set of hypotheses described at the beginning of this chapter, I define the following decision function:
\begin{align*}
    \delta(S_n(X_1,\dots,X_n)) = \begin{cases} 1, & \text{if } S_n(X_1,\dots,X_n) > q; \\ 
    0, & \mbox{otherwise.}\end{cases}
\end{align*}
If the output of $\delta(S_n)$ is $1$, $H_0$ is rejected, if it is $0$, the test fails to reject $H_0$. The exact value of the decision threshold $q$ depends on the test statistic in question as well as on the pre-defined Type I error rate $\alpha$---that is, the false rejection rate of $H_0$---that one is willing to accept on average (see Figure~\ref{fig:error_types} for a visualisation).\par
The goal is then to find the smallest value of $q$ for which the significance level is equal to $\alpha$, that is, 
\begin{alignat*}{3}
    &\phantom{\Longleftrightarrow\quad}\Pr(\delta=1\mid H_0) &= \Pr(S_n(X_1,\dots,X_n) > q\mid H_0) &= \alpha \\
    &\Longleftrightarrow\quad & 1-F_{S_n\mid H_0}(q) &=\alpha \\
    &\Longleftrightarrow\quad & F^{-1}_{S_n\mid H_0}(1-\alpha) &= q
\end{alignat*}
where ${F_{S_n\mid H_0}(\cdot)}$ and ${F^{-1}_{S_n\mid H_0}(\cdot)}$ are the cumulative distribution function and the quantile function, respectively, of $S_n$ under the null hypothesis and ${\Pr(\delta=1\mid H_0)}$ is the probability of the decision function $\delta(\cdot)$ returning $1$ under the null hypothesis. Both $q$ and $\alpha$ are often referred to as the `significance threshold' because they set the threshold for a test statistic or its corresponding cumulative tail probability, respectively, to pass.\par %toadd: maybe add reference to concrete example 

\subsection{The power of a statistical test}
\label{subsec:power}
If a specific significance threshold $\alpha$ is defined, we can calculate the Type II error rate ${\beta = \Pr(\delta = 0\mid H_1)}$ of a test, that is, the probability of not rejecting the null hypothesis in the presence of a real effect (see Figure~\ref{fig:error_types}):
\begin{align*}
    \beta = \Pr(\delta = 0\mid H_1) = \Pr(S_n(X_1,\dots,X_n\mid H_1) < q) = F_{S_n\mid H_1}(q).
\end{align*}
Here, ${F_{S_n\mid H_1}(\cdot)}$ denotes the cumulative distribution function of $S_n$ under the alternative hypothesis. From this, we can calculate the power of a hypothesis test, which is given by ${1 - \beta = 1-F_{S_n\mid H_1}(q)}$. Hence, the power of a test is simply the cumulative probability of a test statistic $S_n$ being larger than its significance threshold, $q_\alpha$ assuming the alternative hypothesis $H_1$ is true.

\myfig{ch2_fig1_error_types}%% filename in figures folder
  {width=\textwidth,height=\textheight}%% maximum width/height, aspect ratio will be kept
  {(A) The probability density functions of two normal distributions with variance $1$ and mean $\mu$ according to the null ($H_0: \mu = 0$) and alternative hypothesis (${H_1: \mu = 2}$), respectively. The critical value $z_{1-\alpha}$ denotes the $(1-\alpha)$-quantile of the standard normal distribution, above which the null hypothesis is rejected. (B) The cumulative density functions of the same two normal distributions are displayed in panel (A). Under $H_0$, the curve crosses the critical value $z_{1-\alpha}$ where ${\Phi(x - \mu_0) = 1-\alpha}$. Under $H_1$, the curve crosses the threshold at ${\Phi(x - \mu_1) = \beta}$.}%% caption %toadd: exact description of figure
  {Type I and Type II errors} %% optional (short) caption for table of figures
  {fig:error_types} %% label

\subsection{The \texorpdfstring{$p$}{p}-value}
\label{subsec:p-value}
For a given observation $x$, the $p$-value is defined as the probability of observing a value that is at least as extreme as $x$ given that the null hypothesis $H_0$ is true. The `extremity' of $x$ is usually assessed by calculating the cumulative probability of randomly drawing values from the null distribution that are at least as far away from the distribution mean as $x$. For one-sided directional hypotheses, the $p$-value is thus defined as
\begin{align*}
    p = \begin{cases} \Pr(X < x \mid H_0), & \text{if testing for inferiority;} \\
    \Pr(X > x \mid H_0), & \text{if testing for superiority.}\end{cases}
\end{align*}
For two-sided non-directional hypotheses---and if the null distribution is symmetric---the $p$-value can simply be calculated by setting ${x = |x|}$ and then doubling the one-sided $p$-value for a superiority test, that is, ${p = 2\cdot \Pr(X > |x| \mid H_0)}$.\par 
If the null distribution is not symmetric, however, the calculation of the two-sided $p$-value does not immediately follow from its definition. One solution is to double the minimum of the $p$-values of both one-sided hypotheses, that is, $$p = 2\cdot\text{ min}(\Pr(X > x \mid H_0), \Pr(X < x \mid H_0)).$$ Another is to `mirror' $x$ around a central tendency $ct$ of the distribution (for example the mean or the median) and then sum up the $p$-values of both one-sided hypotheses:
\begin{align*}
    p = \Pr(X < ct-|x| \mid H_0) + \Pr(X > ct+|x| \mid H_0).
\end{align*}
Given a continuous probability distribution under a simple null hypothesis, the $p$-value is uniformly distributed between $0$ and $1$. This can be shown by simulations or by the following proof\footnote{Note that $p$-values are in fact realisations of a random variable \citep{murdoch_values_2008}. Thus, I will use the uppercase letter $P$ in the following paragraphs to distinguish the random variable from its realisation, even though lowercase letters are usually used for both.}.\par
To show that $P$ is uniformly distributed, I will use the probability integral transform, which states that for a continuous cumulative distribution function $F_X(x)$, the random variable $Y = F_X(X)$ is uniformly distributed as $Y \sim \text{Unif}(0,1)$, that is, ${\Pr(Y < y) = y}$ for ${y \in [0, 1]}$ (see for example \citet[p.~ 54f.]{casella_statistical_2002}).\par
Let ${P = \Pr(X > S_n \mid H_0) = 1-F_{H_0}(S_n)}$ where $F_{H_0}(\cdot)$ denotes the cumulative distribution function of ${S_n \sim \mathcal{N}(0,\sigma^2)}$. Hence, we can write
\begin{align*}
    \Pr(P \leq p) &= \Pr(1-F_{H_0}(S_n) \leq p) = \Pr(F_{H_0}(S_n) \geq 1-p)\\ &=
    \Pr(F_{H_0}^{-1}(F_{H_0}(S_n)) \geq F_{H_0}^{-1}(1-p)) = \Pr(S_n \geq F_{H_0}^{-1}(1-p)) \\
    &= 1-F_{H_0}(F_{H_0}^{-1}(1-p)) = 1-(1-p) \\ 
    &= p
\end{align*}
which concludes the proof.\par
For a composite null hypothesis, e. g. $H_0: \mu < \mu_0 = 0$, uniformity of $P$ only holds with regard to the least favourable value of $\mu$ under the null, that is, $\mu = \mu_0 = 0$. For any $\mu < \mu_0$, the distribution of $P$ is left skewed. Similarly, the distribution of $P$ under a (simple or composite) alternative hypothesis is not uniform either, but right-skewed (see for example \citet{murdoch_values_2008}).

\section{What properties should an evidence measure have?}
\label{sec:evidence_properties}
As mentioned in the previous section, the test statistic $S_n$ depends on the set of hypotheses to be tested and the distributional nature of the data. A test statistic that might be suited for binomially distributed data might lead to false inference if the data are normally distributed. In addition, the values of different test statistics are usually not directly comparable.\par
To alleviate this, we can transform test statistics $S_n$ into an evidence measure $V_n  = h_n(S_n)$ which is also a statistic but for which comparisons are easier to make. \citet[p.~115]{kulinskaya_meta_2008} stated that four desirable properties a one-sided evidence measure $V_n$ should have:
\begin{description}[leftmargin=!,labelwidth=\widthof{\bfseries $\boldsymbol{E_4}$}]
    \item [$\boldsymbol{E_1}$] The one-sided evidence $V_n$ is a monotonically increasing function of the test statistic $S_n$, that is, $h_n(S_{n_1}) \geq h_n(S_{n_2})$ if $n_1 \geq n_2$.
    This ensures that optimal values of $S_n$ are retained by the transformation to $V_n$ and facilitates interpretation.\\
    \item [$\boldsymbol{E_2}$] The distribution of $V_n$ is normal for all values of the unknown parameters. The normal distribution has a variety of desirable statistical properties which facilitate mathematical manipulation and interpretation.\\
    \item [$\boldsymbol{E_3}$] The variance equals one (Var[$V_n$] = $1$) for all values of the unknown parameters. Stabilising the variance to $1$ allows for direct comparison of different evidence measures.\\
    \item [$\boldsymbol{E_4}$] The expected evidence $\tau = \tau(\theta) = \E_{\theta}[V_n]$ is monotonically increasing in $\theta$ from $\tau(0)=0$, that is, larger parameter values $\theta$ indicate larger expectations of the evidence measure $V_n$ and vice versa.
\end{description}
%toadd: why are these properties desirable?
For some combinations of distributions and statistics, such as the $Z$-score based on normally distributed variables (Section~\ref{sec:mean}), all properties $E_1$ to $E_4$ are exactly fulfilled. For others, such as the $Z$-score based on binomially distribute variables (Section~\ref{subsec:z-score}), the properties are only fulfilled asymptotically. Hence, for most combinations of distributions and statistics we need to find a transformation $V_n = h_n(S_n)$, so that $V_n$ fulfils properties $E_1$ to $E_4$ stated above. To fulfil $E_1$, any monotonically increasing function will do. To ensure that $E_2$, $E_3$, and $E_4$ are met, however, slightly more effort is needed.

\subsection{\texorpdfstring{$E_2$}{E2}: Making sure everything is normal}
\label{subsec:normal_transformation}
%toadd: reference to Kulinskaya 126/127 an papers cited there
To meet criterion $E_2$, one usually needs to resort to a distribution-dependent transformation that turns $S_n$ into a normally distributed variable. If $S_n$ follows a lognormal distribution---the simplest case---taking the natural logarithm $\ln(S_n)$ yields a normally distributed variable.\par
If $S_n$ follows a different continuous distribution, an approximate solution can be found by using the transformation $h_n = \phi^{-1}(F_{S_n}(\cdot))$, with $\phi^{-1}$ denoting the quantile function of the standard normal distribution and $F_{S_n}$ denoting the cumulative distribution function of $S_n$. Since $\phi^{-1}$ does not have a closed form description, solutions have to be approximated using numerical methods.\par
Yet another possibility to ensure $E_2$ is to exploit the central limit theorem. For example, if $S_n\sim \text{Bin}(n,p)$, then $S_n \overset{n\rightarrow\infty}{\sim}\mathcal{N}(np,np(1-p))$ (deMoivre-Laplace theorem; see Section~\ref{subsec:z-score} for more details on the usefulness of the central limit theorem in this context).\par

\subsection{\texorpdfstring{$E_3$}{E3}: The variance stabilising transformation}
\label{subsec:vst}
To ensure that property $E_3$ is given, we need to make sure that $h_n$ is variance stabilising, so that $\Var[h_n(S_n)]=1$. If the variance of $S_n$ can be expressed in terms of its expectation passed to a known function $g_n$, that is, ${\Var[S_n] = g_n(\E[S_n])}$, then $h_n$ is defined---up to an additive constant---by
$$h_n(s_n) = \int^{s_n}[g_n(t)]^{-1/2}dt$$
if the indefinite integral exists \citep[p.~126--127]{kulinskaya_meta_2008}. Hence,
\begin{align}
    \{h_n'(\E[S_n])\}^2 = \{g_n(\E[S_n])\}^{-1} = \{\Var[S_n]\}^{-1} \label{eq:vst_1}
\end{align}
For $\Var[S_n]$ to be small enough, we can use a first order Taylor approximation around $\E[S_n]$ to approximate ${\Var[h_n(S_n)] = \Var[V_n]}$. This is done as follows:
\begin{align*}
    V_n &= h_n(\E[S_n]) + (S_n-\E[S_n])h_n'(\E[S_n])+R_1
\end{align*}
Calculating the variance on both sides of the equation yields
\begin{align}
    \Var[V_n] &= \Var[h_n(\E[S_n]) + (S_n-\E[S_n])h_n'(\E[S_n])+R_1] \nonumber \\
    &= \Var[S_n h_n'(\E[S_n])+R_1]  \nonumber \\
    &\simeq \Var[S_n]\{h_n'(\E[S_n])\}^2 \label{eq:vst_2}
\end{align}
Combining Eq.~\ref{eq:vst_1} with Eq.~\ref{eq:vst_2} yields $\Var[V_n] \simeq 1$. Even though finding a variance stabilising function $h_n$ might seem straightforward in theory, it is usually more tedious in practice, because it often depends on unknown parameters, as \citet[p.~127]{kulinskaya_meta_2008} point out.

\subsection{\texorpdfstring{$E_4$}{E4}: Monotonically increasing expectation of evidence}
\label{subsec:monotone_expect_ev}
$E_4$ requires that $\tau(\theta) = \E_{\theta}[V_n]$ is monotonically increasing in $\theta$, starting from $\tau(0) = 0$. In order to find the expectation of $V_n$, we can again resort to a Taylor approximation around $\E[S_n]$, but this time expanding the series up to the second order:
\begin{align*}
    V_n = h_n(\E[S_n]) &+ (S_n-\E[S_n])h_n'(\E[S_n])\\ 
    &+ (S_n-\E[S_n])^2\frac{h_n''(\E[S_n])}{2}+R_2\\
\end{align*}
Taking the expected values on both sides yields
\begin{align*}
    \E[V_n] &= h_n(\E[S_n])+\Var(S_n)\frac{h_n''(\E[S_n])}{2} + R2 \\
    &\simeq h_n(\E[S_n])+\Var(S_n)\frac{h_n''(\E[S_n])}{2}.
\end{align*}

In order to make sure that $\tau(0) = 0$, we can subtract ${h_n(\E[S_n \mid H_0]) = h_n(\theta_0)}$ from $V_n$.

\subsection{The key inferential function}
\label{subsec:kif} %toadd: Check whether kif in passage bleow is correct (especially K(theta-theta_0)
If an evidence statistics $V_n$ fulfils all criteria $E_1$ to $E_4$, it is often possible to rewrite ${\tau(\theta) = \E_{\theta}[V_n-h_n(\theta_0)]}$ as ${\tau(\theta) = \sqrt{n}K_{\theta_0}(\theta)}$, with $K_{\theta_0}$---the so-called `key inferential function' of $\theta$ with respect to a constant $\theta_0$--- independent of the sample size $n$. As outlined in \citet[p.~127--128]{kulinskaya_meta_2008}, $K$ is a useful tool to solve some routine problems which arise in statistical testing, such as:
\begin{description}[leftmargin=!,labelwidth=\widthof{\bfseries $E_4$}]
\item [Sample size calculation:] Recall the hypotheses stated in Section~\ref{subsec:hypothesis_types} in which we wanted to test ${H_0: \theta = \theta_0}$ against ${H_1: \theta > \theta_0}$. To obtain a desired expected evidence $\tau_1$ in favour of a specific $\theta_1$, we need to find 
\begin{align*}
n_1 \geq \blb\frac{\tau_1}{K_{\theta_0}(\theta_1)}\brb^2. \label{eq:sample_size_calculation}
\end{align*}
\item [Power calculation:] The power of a test with significance threshold $\alpha$ and $z_{(1-\alpha)}$ denoting the $(1-\alpha)$-quantile of the standard normal distribution is given by
\begin{align}
    1-\beta &= \Pr(V_n \geq z_{(1-\alpha)}\mid H_1)\\
            &= \Phi(\tau_1 - z_{(1-\alpha)})\\
            &= \Phi(\sqrt{n}\{K_{\theta_0}(\theta_1)\}-z_{(1-\alpha)}). \label{eq:power_calculation}
\end{align}
\item [Confidence intervals:] Confidence intervals can easily be found by using the inverse of the key inferential function $K^{-1}$:
\begin{align}
    \left[K_{\theta_0}^{-1}\blb\frac{V_n-z_{(1-\alpha/2)}}{\sqrt{n}}\brb,K_{\theta_0}^{-1}\blb\frac{V_n-z_{(1-\alpha/2)}}{\sqrt{n}}\brb\right].\label{eq:confidence_interval_calculation}
\end{align}
\end{description}

\section{Statistical evidence in binomial variables}
\label{sec:binomial_var}
Recall the set of hypotheses stated at the beginning of this chapter. Let A and B be two different cancer treatment whose efficacy we want to assess, and let us assume that the primary outcome of interest is the survival rate of the patients; that is, we want to know the proportion of patients $p$ who are still alive twelve months after undergoing one of the two treatments.\par
We can describe the survival of an individual patient after twelve months as a Bernoulli distributed random variable ${L \sim \text{Ber}(p)}$ where $p$ is the survival probability and ${L \in \{0,1\}}$. The null and alternative hypothesis can then be formulated as
\begin{align*}
    H_0 &: p_A \leq p_B \\
    H_1 &: p_A > p_B
\end{align*}
where $p_A$ and $p_B$ are the surviving probabilities of participants receiving treatment A and B, respectively. For the sake of the argument, let us assume that treatment B is already well established and leads to a well-known survival probability $p_B$ twelve months after the treatment. Hence, we only need to estimate $p_A$ to perform our hypothesis test. We can do so by randomly assigning treatment A to $n$ participants and to then count the number of participants $m$ who are still alive twelve months later\footnote{Please note that this example serves illustrative purposes only. It is in no way representative of the way clinical trials are conducted in reality. Both fortunately and unfortunately, real-life clinical trials are more complex endeavours.}. Then, we can use this estimate to perform an exact binomial test.

\subsection{The exact binomial test}
\label{subsec:exact binomial test}
Since the sum of independently and identically distributed $\text{Ber}(p)$-random variables is a binomial random variable $M = \sum_{i=1}^n L_i \sim \text{Bin}(n,p)$, we know that $m$ is a realisation of the binomial random variable ${M \sim \text{Bin}(n,p_A)}$. Hence, the ratio ${\hat{p}_A=m/n}$ serves as an estimate of the twelve-month survival probability of the participants who received treatment A.\par
To assess whether the ratio of surviving participants is greater in the group that received treatment A than in the group that received treatment B, we need a test for statistic $S_n$. In our simplified case, this is simply the total number of surviving participants, hence $S_n(L_1 \dots L_n) = M = \sum_{i=1}^n L_i$.\par
We can now use this statistic together with the following decision function $\delta(\cdot)$ to perform a significance test:
\begin{align*}
    \delta(S_n(L_1,\dots,L_n)) = \begin{cases} 1, & \text{if } S_n(L_1,\dots,L_n) > q_{(B,1-\alpha)}; \\ 0, & \mbox{otherwise.}\end{cases}
\end{align*}
Here, $q_{(B,1-\alpha)}$ denotes the ($1-\alpha$)-quantile of the ${\text{Bin}(n,p_B)}$-distribution. Hence, if the number of surviving participants who received treatment A is larger than the number of surviving participants that would be expected in at least $(1-\alpha)\%$ of the cases in which $n$ patients received treatment B, we would reject $H_0$. This is the so-called exact binomial test. %toadd: add simulations of this test given the example above.

\subsection{Evidence of one-sample binomial tests}
\label{subsec:ev binomial}
%toadd: show that change from 0.5 to 0.55 is significant, but change from 0.7 to 0.75 is not > makes it necessary to stabilise variance in order to improve comparability.
%toadd: add simulations showing this
Testing hypotheses using the exact binomial test has multiple drawbacks, for example: 
\begin{enumerate}
    \item The variance of a random variable following the binomial distribution is dependent on the expected value $p$, which makes the comparison of the results of different test statistics $S_n$ non-trivial. 
    \item The variance of the random variable tends to zero at the extremes of $p$. 
    \item The calculation of the critical values as well as other properties of the test, such as the power, is computationally very costly when compared to other test statistics. 
    \item For each $n$ a different critical value needs to be calculated, even if $H_0$ remains the same.
    \item For large $n$, calculating the cumulative probabilities of a binomial test statistic becomes computationally very taxing.
\end{enumerate}
%tochange/toadd: give explanation why tending to zero is bad (secondly)

Recalling the desirable properties for an evidence measure stated in Section~\ref{sec:evidence_properties}, it is clear the test statistic of the binomial test clearly violates properties $E_2$ and $E_3$, since it follows a binomial distribution and has a non-constant variance for all $S_n$, respectively. In order to make the test statistic fulfil all four properties, we need to transform it. When the sample size $n$ is large enough, we can use the well-known $Z$-score or standard score to do so. 

\subsection{The \texorpdfstring{$Z$}{Z}-score}
\label{subsec:z-score}
The $Z$-score is defined as follows:
$$Z = \frac{X-\mu}{\sigma}$$
If X is a normally distributed random variable with expectation $\mu$ and variance $\sigma^2$, (i. e., $X \sim \mathcal{N}(\mu,\sigma^2)$), the $Z$-score is a random variable following a standard normal distribution ($Z \sim \mathcal{N}(0,1)$), thus fulfilling properties $E_1$ to $E_4$.\par
In order to test a hypothesis, we can simply fix our $\alpha$-level and compare the $Z$-score with the critical values defined as $\Phi^{-1}(1-\alpha)$, where $\Phi^{-1}(\cdot)$ is the quantile function of the standard normal distribution.\par %tochange: currently, the description stated here only holds for hypothesis of the sort: H_0: p1<=p0 H1: p1>p0; need to reformulate to incorporate other cases as well! 
%tochange also: check whether only variance or also mean has to be known
If the sample size $n$ is large enough and by virtue of the central limit theorem (CLT), we can use the $Z$-score to transform the binomial test statistic into a test statistic that fulfils properties $E_1$ to $E_4$. The Lindeberg-Lévy CLT states that if $X_1, \dots X_n$ are identically and independently distributed random variables with E[$X_i$] = $\mu$ and Var[$X_i$] = $\sigma^2 < \infty$, then $\sqrt{n}(\Bar{X}_n-\mu)$ converges in distribution to $\mathcal{N}(0,\sigma^2)$. Hence, 
$$Z_n = \frac{\sqrt{n}(\Bar{X}_n-\mu)}{\sigma} \xrightarrow{d} \mathcal{N}(0,1).$$
In our example from above, we have $\mu = p_B$, $\sigma^2 = \hat{p}_A(1-\hat{p}_A)$ under the null hypothesis, and ${\Bar{X}_n = \Bar{L}_n = \sum_{i=1}^n L_i/n = M/n = \hat{p}_A}$. Hence $$Z_n = \frac{\sqrt{n}(\hat{p}_A-p_B)}{\sqrt{\hat{p}_A(1-\hat{p}_A)}} \xrightarrow[H_0]{d} \mathcal{N}(0,1).$$
If the null hypothesis does not hold, we have
\begin{align}
    Z_n = \frac{\sqrt{n}(\hat{p}_A-p_B)}{\sqrt{\hat{p}_A(1-\hat{p}_A)}} \xrightarrow[H_1]{d} \mathcal{N}(\sqrt{n}\frac{\hat{p}_A-p_B}{\sqrt{\hat{p}_A(1-\hat{p}_A)}},1). \label{eq:Zn_binom}
\end{align}
%toadd: reference to kulinskaya p. 125; toadd: reference to clt; 
However, if $n$ is not large enough or if $p_A$ lies too close to either zero or one, a wide range of problems arises: 
\begin{itemize}
    \item The probability distribution of $Z_n$ becomes leptokurtic with positive skew (${p_A}$ too close to $0$) or negative skew (${p_A}$ too close to $1$ (see Figure~\ref{fig:normal_fit_binomial_MLE} and Figure~\ref{fig:normal_fit_binomial_Ans}).
    \item The empirical mean of $Z_n$ diverges starkly from its theoretical expectation (see Figure~\ref{fig:evidence_binom_MLE} and Figure~\ref{fig:evidence_binom_Ans}).
    \item The empirical variance of $Z_n$ is not stabilised anymore and diverges from $1$ (see Figure~\ref{fig:evidence_binom_MLE} and Figure~\ref{fig:evidence_binom_Ans}).
    \item The Type I error grows beyond the pre-defined $\alpha$-threshold and thus is not controlled anymore (see Figure~\ref{fig:power_binom_MLE} and Figure~\ref{fig:power_binom_Ans}).
    \item The empirical coverage probabilities of the (1-$\alpha$)-confidence intervals lies below the nominal probabilities and deteriorates to $0$ for $p_A$ going to zero or one (see Figure~\ref{fig:CI_binom}). 
\end{itemize}
%These problems generally occur when ${np_A(1-p_A) < 5}$ \citep[p.~140]{kulinskaya_meta_2008}.
In these cases, the normal distribution cannot serve as a good approximation of the variable $M/n$, so it is more prudent to use the key inferential function of the binomial model to transform the binomial test statistic.
%toadd: explanation of figure about power and Type I error

\myfig{ch2_fig2_normal_fit_binomial_MLE}%% filename in figures folder
  {width=\textwidth,height=\textheight}%% maximum width/height, aspect ratio will be kept
  {The standard normal cdf $\Phi$ (black) approximated by $Z_n$ (blue, Eq.~\ref{eq:Zn_binom}) and $V_n$ (red, Eq.~\ref{eq:Vn_binom}) based on binomial variables. The comparison is made for $n=5$ (A), $n=10$ (B), and $n=20$ (C) as well as for three different sets of hypotheses: $H_1: p=0.1$ (solid line), $H_1: p=0.5$ (dashed line), and $H_1: p=0.9$ (dot-dashed line) with $H_0: p=0$ in all three cases. The left column shows cumulative distribution functions, the right column shows quantile-quantile-plots. The approximation is worse for smaller $n$ and for $p_1$ close to either one or zero but $V_n$ outperforms $Z_n$ in all cases. The empirical probability distributions are based on a Gaussian kernel density estimate of $V_n$ and $T_n$ based on 5000 individual draws from a $\text{Bin}(n,p)$-distribution.}%% caption
  {Normal approximation of $Z_n$ and $V_n$ for the binomial model---without continuity correction.}%% optional (short) caption for table of figures
  {fig:normal_fit_binomial_MLE}%% label

\myfig{ch2_fig2_normal_fit_binomial_Ans}%% filename in figures folder
  {width=\textwidth,height=\textheight}%% maximum width/height, aspect ratio will be kept
  {The same description as for Figure~\ref{fig:normal_fit_binomial_MLE} applies. However, normal approximations $Z_n$ and $V_n$ shown here are based on binomial variables and including Anscombe continuity corrections as described in Eq.~\ref{eq:anscombe_correction}.}%% caption %toadd: exact description of figure
  {Normal approximation of $Z_n$ and $V_n$ for the binomial model---with continuity correction.}%% optional (short) caption for table of figures
  {fig:normal_fit_binomial_Ans}%% label

\myfig{ch2_fig3_evidence_binom_MLE}%% filename in figures folder
  {width=\textwidth,height=\textheight}%% maximum width/height, aspect ratio will be kept
  {(Left) Empirical means of $Z_n$ (solid blue, Eq.~\ref{eq:Zn_binom}) and $V_n$ (solid red, Eq.~\ref{eq:Vn_binom}) based on binomial variables compared to theoretical expectations (dashed blue and dashed red, respectively). The comparison is made for $n=5$ (A), $n=10$ (B), and $n=20$ (C) as well as for three different sets of hypotheses: ${H_0: p=0.1}$ (plus), ${H_0: p=0.5}$ (circle), and ${H_0: p=0.9}$ (cross) with ${H_1: p \in [0.01,0.99]}$ in all three cases. (Right) Empirical standard deviations shown for $Z_n$ and $V_n$ for ${H_0: p=0.5}$. The dashed curves indicate the theoretically expected standard deviations. All empirical values are based on $100'000$ independent draws from a binomial distribution. Empirical and theoretical evidence value were weighted by study size for comparison.}%% caption
  {Theoretical and empirical evidence of $Z_n$ and $V_n$ based on binomial variables---without continuity correction.}%% optional (short) caption for table of figures
  {fig:evidence_binom_MLE}%% label

\myfig{ch2_fig3_evidence_binom_Ans}%% filename in figures folder
  {width=\textwidth,height=\textheight}%% maximum width/height, aspect ratio will be kept
  {The same description as for Figure~\ref{fig:evidence_binom_MLE}. However, normal approximations $Z_n$ and $V_n$ shown here are based on binomial variables and including Anscombe continuity corrections as described in Eq.~\ref{eq:anscombe_correction}. The correction clearly improves the approximation of the empirical expectations to their theoretical counterpart and improves the variance stabilisation for both $Z_n$ and $V_n$.}%% caption %toadd: exact description of figure
  {Theoretical and empirical evidence of $Z_n$ and $V_n$ based on binomial variables---with continuity correction.}%% optional (short) caption for table of figures
  {fig:evidence_binom_Ans}%% label

\myfig{ch2_fig4_power_binom_MLE}%% filename in figures folder
  {width=\textwidth,height=\textheight}%% maximum width/height, aspect ratio will be kept
  {(Left) Empirical power curves for ${Z_n > z_{0.95}}$ (blue, Eq.~\ref{eq:Zn_binom}) and ${V_n > z_{0.95}}$ (red, Eq.~\ref{eq:Vn_binom}) based on binomial variables compared to the exact binomial test (black). The comparison is made for $n=5$ (A), $n=10$ (B), and $n=20$ (C) as well as for three different sets of hypotheses: ${H_0: p=0.1}$ (plus), ${H_0: p=0.5}$ (circle), and ${H_0: p=0.9}$ (cross) with ${H_1: p \in [0.01,0.99]}$ in all three cases. In most cases, the blue and red power curves coincide. The dotted line on the horizontal axis denotes $1-\beta = 0.05$. The dotted lines on the vertical axis denote $p_1 = p_0$. (Right) Same curves as in the left column but zoomed in around $1-\beta \in [0,0.1]$. All empirical values are based on $100'000$ independent draws from a binomial distribution.}%% caption
  {Power curves for one-sided superiority tests based on binomial variables---without continuity correction.}%% optional (short) caption for table of figures
  {fig:power_binom_MLE}%% label
  
\myfig{ch2_fig4_power_binom_Ans}%% filename in figures folder
  {width=\textwidth,height=\textheight}%% maximum width/height, aspect ratio will be kept
  {The same description as for Figure~\ref{fig:power_binom_MLE} applies. However, the power curves shown here are based on Anscombe-corrected $Z_n$ and $V_n$ values as described in Eq.~\ref{eq:anscombe_correction}. The correction does in general not improve the power or the control of the Type I error and even increases the Type I error for $V_n$ when $n$ and $p_1$ are small.}%% caption
  {Power curves for one-sided superiority tests based on binomial variables---with continuity correction.}%% optional (short) caption for table of figures
  {fig:power_binom_Ans}%% label  

\myfig{ch2_fig5_CI_binom}%% filename in figures folder
  {width=\textwidth,height=\textheight}%% maximum width/height, aspect ratio will be kept
  {(Left) Empirical coverage probabilities for nominal $95\%$ confidence intervals around $Z_n$ (blue, Eq.~\ref{eq:Zn_binom}) and $V_n$ (red, Eq.~\ref{eq:Vn_binom}) based on binomial variables. The comparison is made for $n=5$ (A), $n=10$ (B), $n=30$ (C), and $n=50$ (D) and for ${H_0: p=0.5}$ (circle) with ${H_1: p \in [0.01,0.99]}$. The dotted black line on the horizontal axis denotes a nominal coverage probability of $95\%$. (Right) Same comparisons shown as in the left column but for Anscombe-corrected $Z_n$ and $V_n$ as described in Eq.~\ref{eq:anscombe_correction}. All values are based on $100'000$ independent draws from a binomial distribution.}%% caption
  {Empirical coverage probabilities of $95\%$ confidence intervals around $Z_n$ and $V_n$.}%% optional (short) caption for table of figures
  {fig:CI_binom}%% label

\subsection{The key inferential function for the binomial model}
\label{subsec:kif binom}
%toadd: reference Kulinskaya et al. 2008, 140
For the set of hypotheses $H_0: p_A \leq p_B$ and $H_1: p_A > p_B$ with $p_B$ assumed to be known, the key inferential function for the binomial model is given by
$$K_{p_B}(p_A) = 2\{\text{sin}^{-1}(\sqrt{p_A})-\text{sin}^{-1}(\sqrt{p_B})\}.$$
The expectation of the transformed test statistic is then $\E[V_n] = \sqrt{n}K_{p_B}(p_A)$ (see \citet[p.~139--140]{kulinskaya_meta_2008} for derivation) and the variance stabilised test statistic $V_n$ is given by
\begin{align}
    V_n = 2\sqrt{n}\{\text{sin}^{-1}(\sqrt{\hat{p}_A})-\text{sin}^{-1}(\sqrt{p_B})\}. \label{eq:Vn_binom}.
\end{align}
Simulations have shown that the empirical mean and variance of $V_n$ are much closer to their theoretical counterparts than it is the case for $Z_n$ (see Figure~\ref{fig:evidence_binom_MLE}). In addition, confidence intervals for $V_n$ are much closer to their nominal levels than it is the case for $Z_n$ (see Figure~\ref{fig:CI_binom}). However, $V_n$ is in general not better at controlling the Type I error: As can be seen in Figure~\ref{fig:power_binom_MLE}, the power curves of $Z_n$ and $V_n$ coincide in most scenarios with $V_n$ outperforming $Z_n$ only for very small $n$ and $p_1$.\par
For both $V_n$ and $Z_n$, the normal approximation can be improved by applying the Anscombe continuity correction \citep[p.~140--141]{kulinskaya_meta_2008}:
\begin{align}
    \tilde{p}_A = (M+3/8)/(n+3/4) \label{eq:anscombe_correction}   
\end{align}
This can also be seen visually by comparing Figure~\ref{fig:normal_fit_binomial_MLE} (no continuity correction) to Figure~\ref{fig:normal_fit_binomial_Ans} (Anscombe correction) and---even more clearly---by comparing Figure~\ref{fig:evidence_binom_MLE} (no continuity correction) to Figure~\ref{fig:evidence_binom_Ans} (Anscombe correction).\par
The Anscombe correction also improves the empirical coverage probability of nominal confidence intervals for both $Z_n$ and $V_n$ (see Figure~\ref{fig:CI_binom}), but does not help with controlling the Type I error (see Figure~\ref{fig:power_binom_Ans}). Notably, for very small $n$ and $p_1$, the Type I error is greater when using the Anscombe-corrected $V_n$ as test statistic.
%toremove/Toadd: further improves approximate normality of $V_n$ which holds for all $n$ such that ${np(1-p)\geq 5}$. Conversely, my simulations have shown that approximate normality of $Z_n$ only holds for SIMULATIONTOADD.
%toadd: criterion at which approximate normality holds; idea: could us the shapiro Wilk test to test for normalit at Alpha = 0.01.
%toadd: also add range of variance stability for both z-score and vst
%toadd: explanation of what I did in "TheoreticalSimulations.R", including RSS metric to test for normality
\clearpage
\section{Evidence for the difference in means}
\label{sec:mean}
Instead of choosing the proportion of surviving patients after a given time period as the primary outcome, another measure to assess the efficacy of a treatment could be the survival time in months after receiving the treatment.\par 
To assess whether treatment A is in fact superior to treatment B, we would randomly and independently assign patients to one of the two treatments and measure their survival time. Let us furthermore assume that the survival times $Y_A$ and $Y_B$ of patients in treatment group A and B, respectively, are log-normally distributed\footnote{See for example \citet{royston_lognormal_2001} or \citet{chapman_innovative_2013} for cases in which this assumption can be justified}, that is, ${Y_A \sim \text{Lognormal}(\mu_A, \sigma_A^2)}$ and ${Y_B \sim \text{Lognormal}(\mu_B, \sigma_B^2)}$.\par

By log transforming $Y_A$ and $Y_B$ we get two normally distributed variables ${X_A = \text{ln}(Y_A) \sim \mathcal{N}(\mu_A, \sigma_A^2)}$ and $X_B = \text{ln}(Y_B) \sim \mathcal{N}(\mu_B, \sigma_B^2)$. For testing, our hypotheses need to be reformulated as %toadd: correct notation
\begin{align*}
    H_0 &: \mu \leq \mu_0 \\
    H_1 &: \mu > \mu_0
\end{align*}
with $\mu = \mu_A-\mu_B$ and $\mu_0 = 0$. To test these hypotheses, we define our test statistic as $$S_n = \Bar{X}_{n_A}-\Bar{X}_{n_B}.$$
$\Bar{X}_{n_A}$ and $\Bar{X}_{n_B}$ represent the sampled arithmetic mean of the log transformed survival time of $n_A$ and $n_B$ patients receiving treatment A and B, respectively. Both means are normally distributed with ${\Bar{X}_{n_A}\sim \mathcal{N}(\mu_A,\sigma^2/n_A)}$ and ${\Bar{X}_{n_B}\sim \mathcal{N}(\mu_B,\sigma^2/n_B)}$. The difference of these two means is normally distributed as well, namely ${(\Bar{X}_A-\Bar{X}_B) \sim \mathcal{N}(\mu_A-\mu_B, \sigma_A^2/n_A+\sigma_B^2/n_B)}$. The corresponding decision function is then
\begin{align*}
    \delta(S_n(\Bar{X}_{n_A},\Bar{X}_{n_B})) = \begin{cases} 1, & \text{if } S_n(\Bar{X}_{n_A},\Bar{X}_{n_B}) > q_{H_0,1-\alpha}; \\ 0, & \mbox{otherwise;}\end{cases}
\end{align*}
with the threshold value $q_{(H_0,1-\alpha)}$ denoting the $(1-\alpha)$-quantile of a normal distribution with mean ${(\mu_A-\mu_B) = \mu_0 = 0}$ and variance ${\sigma_A^2/n_A+\sigma_B^2/n_B}$ under the null hypothesis.\par
This statistic fulfills properties $E_1$, $E_2$, and $E_4$, but has Var[$S_n] \neq 1$. If we know the variance ${\sigma^2(1/n_A+1/n_B)}$, we can again use the $Z$-statistic
$$Z_n = \frac{S_n-\mu_0}{\sigma_{(n_A+n_B)}} = \frac{(\Bar{X}_A-\Bar{X}_B)-\mu_0}{\sigma\sqrt{(1/n_A+1/n_B)}}\overset{H_0}{\sim} \mathcal{N}(0,1),$$ 
described in Section \ref{subsec:z-score} to stabilise the variance of the test statistic, thereby fulfilling property $E_3$ as well.\par

\subsection{Student's \texorpdfstring{$t$}{t}-statistic}
\label{subsec:student's t-statistic}
%toadd: explanatation & proof: https://math.stackexchange.com/questions/2189374/showing-that-s-n2-converges-to-sigma-2-in-probability
If the population variance $\sigma^2$ is unknown it can be estimated by the empirical variance $${s_n^2 = 1/(n-1) \sum_{i=1}^n (x_i-\bar{x})^2}.$$ Plugging this into the $Z$-statistic yields the so-called Student's $t$-statistic:
\begin{align}
    T_n = \frac{\ssqrt{n}(\bar{X}_n-\mu_0)}{s_n}.\label{eq:Tn_stud}
\end{align}
In the specific case of the hypotheses stated at the beginning of this section, the $t$-statistic is given by
$$T_n = \frac{\Bar{X}_{N}-\mu_0}{s_{N}(1/n_A+1/n_B)} \sim t(\nu = N-2)$$
with
$$N = n_A+n_b,\quad \Bar{X}_{N}=\Bar{X}_A-\Bar{X}_B,\quad s_{N} = \sqrt{\frac{(n_A-1)s_{n_A}^2+(n_B-1)s_{n_B}^2}{(n_A-1)+(n_B-1)}}.$$
$s_{N}$ is called the pooled sample variance and serves to fix the variance of the test statistic to 1. Note that this assumes that the true variance $\sigma^2$ is the same across both populations. If this is not the case, it is better to use Welch's generalisation of the $t$-test \citep{welch_generalization_1947}. \par As in the examples outlined in Section~\ref{sec:binomial_var}, I am going to treat the distributional parameters of treatment B as known and thus will use the $t$-statistic described in Eq.~\ref{eq:Tn_stud} for the following calculations and simulations. 

Since $s_n^2$ converges to $\sigma^2$ in probability, we can treat $\sigma \simeq s_n$ and $T_n \overset{H_0}{\simeq} Z_n$ and just use the $Z$-test. However, if $n$ is not large enough or if $\mu_1$ deviates too much from $\mu_0$ the normal approximation of $T_n$ does not hold anymore and problems similar to those described in Section~\ref{subsec:z-score} appear: 
\begin{itemize}
    \item The probability distribution of $T_n$ becomes leptokurtic with a positive skew (${\mu_1 > \mu_0}$) or negative skew (${\mu_1 < \mu_0}$ with symmetry only attained if ${\mu_1 = \mu_0}$ (see Figure~\ref{fig:normal_fit_student_MLE} and Figure~\ref{fig:normal_fit_student_Corr}).
    \item The empirical mean of $T_n$ diverges from its theoretical expectation assuming normality (see Figure~\ref{fig:evidence_student_MLE} and Figure~\ref{fig:evidence_student_Corr}).
    \item The empirical variance of $T_n$ is not stabilised anymore and diverges from $1$ (see Figure~\ref{fig:evidence_student_MLE} and Figure~\ref{fig:evidence_student_Corr}).
    \item The Type I error grows beyond the pre-defined $\alpha$-threshold and thus is not controlled anymore (see Figure~\ref{fig:power_student_MLE} and Figure~\ref{fig:power_student_Corr}).
    \item The empirical coverage probabilities of the (1-$\alpha$)-confidence intervals lies clearly below the nominal probabilities and deteriorates further with $\mu_1$ diverging from $\mu_0$ (see Figure~\ref{fig:CI_student}. 
\end{itemize}
For small sample sizes it is therefore inappropriate to treat $T_n$ as normally distributed to test for differences in means. Instead, we should apply the variance stabilised test statistic $V_n$ based on Student's $t$-distribution.\par 
If $X$ is a random variable with normal distribution, $T_n$ is a random variable with a central Student's $t$-distribution and ${\nu = n-1}$ degrees of freedom denoted as $Tn \sim t(\nu)$ \citep{student_probable_1908}, hence the decision function of the so-called $t$-test is equal to
\begin{align*}
    \delta(T_n(X_1,\dots X_n) = \begin{cases} 1, & \text{if } T_n(X_1,\dots, X_n) > q_{(H_0,1-\alpha)}; \\ 0, & \mbox{otherwise;}\end{cases}
\end{align*}
with $q_{H_0,1-\alpha}$ equal to the $(1-\alpha)$-quantile of the $t(\nu)$-distribution.\par 
In order to calculate the evidence for a set of hypotheses, we also need to know the distribution of $T_n$ under the alternative hypothesis $H_1: \mu > \mu_0$. To do so, we can rewrite $T_n$ as 
$$T_n = \frac{\ssqrt{n}(\Bar{X}_n-\mu) + \ssqrt{n}(\mu-\mu_0)}{s_n}$$
which follows a noncentral $t$-distribution with $\nu = n-1$ degrees of freedom and non-centrality parameter $\lambda = \sqrt{n}(\mu-\mu_0)/\sigma$, that is, ${T_n \sim t(\lambda,\nu)}$ \citep[p.~159--160]{kulinskaya_meta_2008}.\par %toadd: maybe also add proof why non-central t
\subsection{The key inferential function for the \texorpdfstring{$t$}{t}-statistic}
\label{subsec:kif_t-statistic}
As before with the binomial test statistical, we can transform the $t$-statistic into an evidence measure $V_n$. To do so, we can apply the following transformation function (see \citet[p.~160--161]{kulinskaya_meta_2008} for derivation):
$$h_n(T_n) = \ssqrt{2(n-1)} \text{sinh}^{-1}\blb T_n/\sqrt{2(n-1)}\brb \simeq \ssqrt{2n} \text{sinh}^{-1}\blb T_n/\sqrt{2n}\brb$$ %toadd: derivation of vst and key inferential function of student's t distribution. p160 & 161
The evidence $V_n$ in a $t$-statistic for testing ${H_0: \mu \leq \mu_0}$ against ${H_1: \mu > \mu_0}$ is therefore given by
\begin{align}
    V_n = \ssqrt{2n}\text{sinh}^{-1}\blb\frac{T_n}{\sqrt{2n}}\brb=\ssqrt{2n}\text{sinh}^{-1}\blb\frac{(\bar{X}_n-\mu_0)/s_n}{\sqrt{2}}\brb \label{eq:Vn_stud}
\end{align}
%toadd: finite sample corrected evidence $T_{\text{corrected}}=(\frac{n-1.7}{n-1})\sqrt{2n}sinh^{-1}(S_n/\sqrt{2n})$ improves corrected evidence in tails\\
%toadd: is evidence normally distributed? (Laubscher) and under which circumbstances
and its expectation is
$${\E[V_n] = \tau(\mu) = \ssqrt{n} \ssqrt{2} \text{sinh}^{-1}\blb\frac{\mu-\mu_0}{\sqrt{2}}\brb =  \ssqrt{n} K_{\mu_0}(\mu)}.$$
Simulations confirm that the normal approximation and variance stabilisation are better for $V_n$ than for $T_n$, especially if $\mu_1 \neq \mu_0$ (see Figure~\ref{fig:normal_fit_student_MLE}). The same holds true for the correspondence of the empirical mean and variance to their theoretical counterparts (see Figure~\ref{fig:evidence_student_MLE}).\par 
Whereas $V_n$ does not completely control Type I error rates when $n$ is low but is consistently better than $T_n$ when using standard normal quantiles as critical values. Similarly, the empirical coverage probability of the confidence intervals around $V_n$ lies below the nominal probability for small $n$ but consistently outperforms the empirical coverage probability of confidence intervals around $T_n$ when using using standard normal quantiles.\par

The normal approximation and performance of both $V_n$ and $Z_n$ can be improved by applying the following finite sample correction \citep[p.~161]{kulinskaya_meta_2008}:
\begin{align}
    V_n^* = \blb\frac{n-1.7}{n-1}\brb \ssqrt{2n}\text{sinh}^{-1}\blb\frac{T_n}{\sqrt{2n}}\brb \label{eq:finite_sample_correction}
\end{align} %toadd: Finite sample correctino for Z_n
The improvement in normal approximation and variance stabilisation is best in the tails of the distribution and is also observable visually by comparing Figure~\ref{fig:normal_fit_student_MLE} (no correction applied) to Figure~\ref{fig:normal_fit_student_Corr} (including finite sample correction) as well as by comparing Figure~\ref{fig:evidence_student_MLE} (no correction applied) to Figure~\ref{fig:evidence_student_Corr} (including finite sample correction).\par
The finite sample correction also helps to control the Type I error. The corrected $T_n$ meets the nominal $\alpha$-threshold almost, the corrected $V_n$ meets it completely, even for very small $n$ (see Figure~\ref{fig:power_student_Corr}). Finally, the correction does also markedly improve the empirical coverage probability of confidence intervals around $V_n$ and $T_n$ (see Figure~\ref{fig:CI_student}).
%TOADD
%TOADD:ed: We expect to observe more low significant p values (p < .01) than high significant p values (.04 < p < .05; Cumming, 2008; Hung,O’Neill, Bauer, & Kohne, 1997; Simonsohn et al., 2014; Wallis, 1942). F
%toadd: proof that distribution is left skewed for values of mu < mu0
%to add: simulation showing distribution of p-values

%next steps:
%apply example to z and t-test
%finish theoretical description of z and t-test
%start writing theoretical part about Evidence and vst

\myfig{ch2_fig6_normal_fit_student_MLE}%% filename in figures folder
  {width=\textwidth,height=\textheight}%% maximum width/height, aspect ratio will be kept
  {The standard normal cdf $\Phi$ (black) approximated by $T_n$ (blue, Eq.~\ref{eq:Tn_stud}) and $V_n$ (red, Eq.~\ref{eq:Vn_stud}) based on normally distributed variables with variance ${\sigma^2 = 4}$. The comparison is made for $n=5$ (A), $n=10$ (B), and $n=20$ (C) as well as for three different sets of hypotheses: ${H_1: \mu = -2}$ (solid line), ${H_1: \mu = 0}$ (dashed line), and ${H_1: \mu = 2}$ (dot-dashed line) with ${H_0: \mu=0}$ in all three cases. The left column shows cumulative distribution functions, the right column shows quantile-quantile-plots. The approximation is worse for smaller $n$ and for $mu_1$ farther away from $\mu_0$ but $V_n$ outperforms $Z_n$ in all cases. The empirical probability distributions are based on a Gaussian kernel density estimate of $V_n$ and $T_n$ based on 5000 individual draws from a $\mathcal{N}(\mu_1,4)$-distribution.}%% caption
  {Normal approximation of $T_n$ and $V_n$ for the difference in normally distributed means---without finite sample correction.}%% optional (short) caption for table of figures
  {fig:normal_fit_student_MLE}%% label

\myfig{ch2_fig6_normal_fit_student_Corr}%% filename in figures folder
  {width=\textwidth,height=\textheight}%% maximum width/height, aspect ratio will be kept
  {The same description as for Figure~\ref{fig:normal_fit_student_MLE} applies. However, $Z_n$ and $V_n$ shown here include the finite sample correction described in Eq.~\ref{eq:finite_sample_correction}.}%% caption %toadd: exact description of figure
  {Normal approximation of $T_n$ and $V_n$ for the difference in normally distributed means---with finite sample correction.}%% optional (short) caption for table of figures
  {fig:normal_fit_student_Corr}%% label

\myfig{ch2_fig7_evidence_student_MLE}%% filename in figures folder
  {width=\textwidth,height=\textheight}%% maximum width/height, aspect ratio will be kept
  {(Left) Empirical means of $T_n$ (solid blue, Eq.~\ref{eq:Tn_stud}) and $V_n$ (solid red, Eq.~\ref{eq:Vn_stud}) based on normally distributed variables with variance ${\sigma^2 = 4}$ compared to theoretical expectations (dashed blue and dashed red, respectively). The comparison is made for $n=5$ (A), $n=10$ (B), and $n=20$ (C) as well as for three different sets of hypotheses: ${H_0: \mu = -2}$ (plus), ${H_0: \mu=0}$ (circle), and ${H_0: \mu=2}$ (cross) with ${H_1: \mu \in [-2,2]}$ in all three cases. (Right) Empirical standard deviations shown for $T_n$ and $V_n$ for ${H_0: \mu=0}$. The dashed curves indicate the theoretically expected standard deviations. All empirical values are based on $100'000$ independent draws from a $\mathcal{N}(\mu_1,4)$-distribution. Empirical and theoretical evidence value were weighted by study size for comparison.}%% caption
  {Theoretical and empirical evidence of $T_n$ and $V_n$ for the difference in normally distributed means---without finite sample correction.}%% optional (short) caption for table of figures
  {fig:evidence_student_MLE}%% label

\myfig{ch2_fig7_evidence_student_Corr}%% filename in figures folder
  {width=\textwidth,height=\textheight}%% maximum width/height, aspect ratio will be kept
  {The same description as for Figure~\ref{fig:evidence_student_MLE}. However, $Z_n$ and $V_n$ shown here include the finite sample correction described in Eq.~\ref{eq:finite_sample_correction}. The correction clearly improves the approximation of the empirical expectations to their theoretical counterpart and improves the variance stabilisation for both $T_n$ and $V_n$, especially in the tails.}%% caption %toadd: exact description of figure
  {Theoretical and empirical evidence of $T_n$ and $V_n$ for the difference in normally distributed means---with finite sample correction.}%% optional (short) caption for table of figures
  {fig:evidence_student_Corr}%% label

\myfig{ch2_fig8_power_student_MLE}%% filename in figures folder
  {width=\textwidth,height=\textheight}%% maximum width/height, aspect ratio will be kept
  {(Left) Empirical power curves for ${T_n > z_{0.95}}$ (blue) and ${V_n > z_{0.95}}$ (red) compared to ${T_n > t_{(n-1,0.95)}}$ (black). The comparison is made for $n=5$ (A), $n=10$ (B), and $n=20$ (C) as well as for three different sets of hypotheses: $H_0: \mu=-2$ (plus), ${H_0: \mu=0}$ (circle), and ${H_0: \mu=2}$ (cross) with ${H_1: \mu \in [-2,2]}$ in all three cases. In most cases, the blue and red power curves coincide. The dotted line on the horizontal axis denotes ${1-\beta = 0.05}$. The dotted lines on the vertical axis denote $mu_1 = mu_0$. (Right) Same curves as in the left column but zoomed in around ${1-\beta \in [0,0.1]}$. $V_n$ and $T_n$ are calculated according to Eq.~\ref{eq:Vn_stud} and Eq.~\ref{eq:Tn_stud}, respectively, based on $100'000$ independent draws from a $\mathcal{N}(\mu_1,4)$-distribution.}%% caption
  {Power curves for one-sided superiority tests based on difference in normally distributed means---without finite sample correction.}%% optional (short) caption for table of figures
  {fig:power_student_MLE}%% label
  
\myfig{ch2_fig8_power_student_Corr}%% filename in figures folder
  {width=\textwidth,height=\textheight}%% maximum width/height, aspect ratio will be kept
  {The same description as for Figure~\ref{fig:power_student_MLE} applies. However, the power curves shown here include the finite sample correction described in Eq.~\ref{eq:finite_sample_correction}. The correction clearly improves control of Type I error rates for both $T_n$ and $V_n$ with the latter meeting the nominal $\alpha$-threshold.}%% caption
  {Power curves for one-sided superiority tests based on difference in normally distributed means---with finite sample correction.}%% optional (short) caption for table of figures
  {fig:power_student_Corr}%% label  

\myfig{ch2_fig9_CI_student}%% filename in figures folder
  {width=\textwidth,height=\textheight}%% maximum width/height, aspect ratio will be kept
  {(Left) Empirical coverage probabilities for nominal $95\%$ confidence intervals around $T_n$ (blue) and around $V_n$ (red) using standard normal quantiles. The comparison is made for $n=5$ (A), $n=10$ (B), $n=30$ (C), and $n=50$ (D) and for ${H_0: \mu=0}$ (circle) with ${H_1: \mu \in [-2,2]}$. The dotted black line on the horizontal axis denotes a nominal coverage probability of $95\%$. (Right) Same comparisons shown as in the left column but with $T_n$ and $V_n$ including the finite sample correction described in Eq.~\ref{eq:finite_sample_correction}. $V_n$ and $T_n$ are calculated according to Eq.~\ref{eq:Vn_stud} and Eq.~\ref{eq:Tn_stud}, respectively, based on $100'000$ independent draws from a $\mathcal{N}(\mu_1,4)$-distribution.}%% caption
  {Empirical coverage probabilities of $95\%$ confidence intervals around $T_n$ and $V_n$.}%% optional (short) caption for table of figures
  {fig:CI_student}%% label
\chapter{Detecting and Correcting Publication Bias}
\label{cha:publication bias}
\epigraph{\centering \textit{`The political principle that anything can be proved by statistics arises from the practice of presenting only a selected sub-set of the data available.'}}{--- Ronald A. Fisher,\\Statistical Methods and Scientific Induction, (1955, p.~75)}

Aggregating evidence measures from different studies is quite simple---theoretically. Given access to the raw data of each study and assuming that each study was designed and conducted in the same manner, one can simply calculate summary and test statistics based on the total aggregate of the data. Of course, statistical practice is hardly ever so straightforward.\par

Firstly, easy access to indicative summary statistics such as mean, standard deviation and sample size, let alone the raw data, is still very rare. Secondly, study designs and protocols often deviate heavily between different sites even if the intention is to measure the same primary outcome. Thirdly, even if access to raw data and consistent and rigorous study designs and protocols are given, attempts to find accurate global evidence measures might be hampered since the available body of literature can be biased in favour of certain study properties other than its methodological quality. \par

This is not a particularly novel insight. \citet{sterling_publication_1959} already pointed out in 1959 that `when a fixed level of significance is used as critical criterion for selecting reports for dissemination in professional journals, it may result in embarrassing and unanticipated results'. One of these `embarrassing results' might be that the majority of journals are `filled with the 5\% of the studies that show Type I errors' while the other 95\% of studies with non-significant test results remain largely unpublished \citep{rosenthal_file_1979}.\par 

To make things worse, selection for statistical significance is by far not the only reason for the occurrence of this so-called `publication bias'. Other properties that might influence the publication probability of a result include its novelty \citep{auspurg_what_2011}, its concordance with prior knowledge \citep{cooper_finding_1997}, its newsworthiness \citep{auspurg_what_2011}, its economic value \citep{chalmers_minimizing_1990} or its political content \citep{eitan_research_2018}. In addition, other outcomes of the same study \citep{dickersin_existence_1990}, such as the funding source of said study \citep{dickersin_existence_1990}, economic or ideological conflicts of interests \citep{chalmers_minimizing_1990, eitan_research_2018}, ignorance about previous studies \citep{chalmers_minimizing_1990} and even the motivation \citep{chalmers_minimizing_1990, cooper_finding_1997,auspurg_what_2011, franco_publication_2014} or reputation \citep{dickersin_existence_1990} of the study authors have been proposed as potential influences on the probability of publication.\par

However, since \textit{post hoc} correction of publication bias is only possible if the probability of publishing a study is influenced by the statistical properties of the result, this chapter will exclusively focus on methods to detect and---at least theoretically---correct for publication bias arising from using a fixed level of significance as critical criterion for publication.

\section{Significance: The fickle gatekeeper of scientific publishing}
\label{sec:significance_fickle}
Using a pre-defined threshold with which to compare the outcome of a statistical test has been commonplace since the early days of modern statistical inference \citep{cowles_origins_1982}. William Sealy Gosset wrote already in 1908 that a deviation of `three times the probable error in the normal curve [...] would be considered significant [for most purposes]' \citep[p.~13]{student_probable_1908}.\par

Three times the probable error roughly corresponds to two standard deviations of a normal distribution or an $\alpha$-level of 0.05---a threshold that was repeatedly adopted by \citet{fisher_statistical_1925} in his seminal book `Statistical Methods for Research Workers' to assess whether a particular result warrants further investigation. Fisher called it `convenient to take this point as a limit in judging whether a deviation [from the null] is to be considered significant or not' (p.~45), but simply used this threshold simply as a rough filter and never as a definitive decision rule. `If one in twenty does not seem high enough odds, we may, if we prefer it, draw the line at one in fifty [...], or one in a hundred [...]. Personally, the writer prefers to set a low standard of significance at the 5 per cent point, and ignore entirely all results which fail to reach this level. A scientific fact should be regarded as experimentally established only if a properly designed experiment \textit{rarely fails} to give this level of significance. The very high odds sometimes claimed for experimental results should usually be discounted, for inaccurate methods of estimating error have far more influence than has the particular standard of significance chosen' \citep[p.~85--86]{fisher_arrangements_1926}.\par

Similarly, the hypothesis testing approach developed by Jerzy Neyman and Egon Pearson hinged on fixing a `level of rejection' \citep{neyman_testing_1933} or `critical region' \citep{neyman_problem_1933}. They also made abundantly clear that the exact choice of these significance levels should be instructed by the researcher's assessment of how damaging the false rejection of the null hypothesis would be. `In making a decision upon which subsequent action will be based we are influenced by the consequences which follow from a wrong decision; some errors will matter more than others' \citep[p.~509]{neyman_testing_1933}.

Nevertheless, most users of statistics quickly started to treat significance thresholds as rigid decision boundaries with which to distinguish `good' from `bad' results. `Significance' has in many ways become a synonym for `relevance', `importance', or even `scientific quality'. It was shown that even journal reviewers tend to assess the methodological quality of a study more favourably if it reports significant findings \citep{mahoney_publication_1977}, so it should not come as a surprise either that statistically significant studies are heavily over-represented in publications across a wide range of disciplines and generally have a higher probability of being submitted and considered for publication in the first place \citep{hedges_estimation_1984, begg_publication_1988, dickersin_existence_1990, easterbrook_publication_1991, cooper_finding_1997, gerber_testing_2001, dickersin_publication_2005, gerber_can_2006, ioannidis_exploratory_2007, gerber_publication_2008, weiss_identification_2011, fanelli_negative_2012, franco_publication_2014, kuhberger_publication_2014, flint_there_2015, berning_publication_2016}.\par
%TOADD: Within the social sciences, Sterling (1959) was the first to conduct an audit of the four main psychology journals and found that in a single year, 97.3% of articles rejected the null hypothesis at the 5% significance level. In a recent audit of the same journals, Sterling et al. (1995) note that the situation had changed little since the 1950s, with 95.6% of articles rejecting the null. Surveys of psychologists have found that authors are reluctant to submit insignificant findings and journal editors are less likely to publish them (Greenwald 1975). Coursol and Wagner (1986) find that psychologists reported submitting 82.2% of studies that had significant findings but only 43.1% of studies with neutral or negative findings. In addition to this bias in 6 the submission stage, researchers reported that 65.9% of significant studies submitted for publication were ultimately accepted whereas only 21.5% of insignificant studies eventually appeared in print.
These distorting effects on scientific research in general and scientific publishing in particular have sparked heated debates among statisticians about the role of significance for statistical inference, with some scholars calling for abandoning the  usage  of  statistical significance as decision criterion altogether \citep{mcshane_abandon_2019, amrhein_retire_2019} and others defending the merits of predefined decision rules such as significance tests \citep{ioannidis_importance_2019}.

There is consensus, however, that an exaggerated fixation on statistically significant results or `significosis', as \citet{antonakis_doing_2017} dubbed it, can heavily bias meta-analyses of the published body of literature by inflating effect sizes and distorting the accuracy of estimates. The following sections are therefore dedicated to methods for the detection and correction of publication bias.

\section{Detecting publication bias: How much significance is too much?}
\label{sec:detect_publication_bias}
Even though an increased amount of significant findings in the research literature should definitely give meta-analysts pause, it cannot be taken as proof for the existence of publication bias. After all, researchers usually do not embark haphazardly on scientific endeavours but rather base their studies on prior knowledge about which ideas are worth pursuing. Hence, in the ideal world of carefully designed and executed experiments, one would expect researchers to find real effects (as opposed to statistical blips) on a regular basis, therefore increasing the relative amount of statistically significant findings in the published literature. Hence, it is important to have methods with which to distinguish over-representation of significant findings in the literature due to publication bias from over-representation because of real effects.\par

\subsection{The file drawer problem}
\label{subsec:file drawer}
%Add Hedges (which year) with regard to distribution of effect size under the null.
\citet{rosenthal_file_1979} offered a crude, but straightforward way to calculate the number of studies that theoretically got stuck in the so-called `file drawer' in a worst-case scenario; that is, a scenario in which all published inferences about hypotheses consist exclusively of Type I errors.\par

To estimate how many unpublished studies would be needed to bring the overall $p$-value of all studies combined (both published and unpublished) to a certain significance level, Rosenthal suggested combining all published results by using the standard normal deviates $z_j$ associated with the $p$-values of each result $j$ so that $$z_c = k\bar{z}_k/\sqrt{k} = \sqrt{k}\bar{z}_k.$$
Here, $\bar{z}_k$ denotes the arithmetic mean of all standard normal deviates combined. Since ${Z_j \sim \mathcal{N}(0,1)}$, it holds that ${\frac{1}{k}\sum_{i=1}^k Z_j = \bar{Z}_k \sim \mathcal{N}(0,1/k)}$ and ${\sqrt{k}\bar{Z}_k \sim \mathcal{N}(0,1)}$, hence ${z_c}$ is the realisation of a standard normally distributed variable. By assuming that all $k$ studies are independent from each other, we can calculate the number of studies $o$ needed to achieve the desired significance threshold $\alpha$ with standard normal quantile $z_{(1-\alpha)}$ for a one-sided superiority test:
\begin{align*}
    \phantom{\Longleftrightarrow}\quad z_{(1-\alpha)} &= \frac{k\bar{z}_k + x\bar{z}_o}{\sqrt{k+o}}\\
    \Longleftrightarrow\quad x &= \frac{(k\bar{z}_k + o\bar{z}_o)^2}{z_{(1-\alpha)}^2}-k.
\end{align*}
If one assumes---as Rosenthal did---that the omitted studies $o$ show a null effect on average ($\bar{z}_o = 0$), the expression can be simplified to
\begin{align*}
    o = \frac{(k\bar{z}_k)^2}{z_{(1-\alpha)}^2}-k.
\end{align*}
However, this heavily overestimates the number of potentially omitted studies because the mean of the standard normal distribution truncated to the right at ${z_{(1-\alpha)} < \infty}$ is negative. It is therefore better to calculate $o$ by using ${\bar{z}_o = \E[Z \mid Z < z_{(1-\alpha)}]}$ which leads to
\begin{align*}
    o^*= \frac{-2k \bar{z}_k \bar{z}_o + z_{(1-\alpha)}^2 - 
    z_{(1-\alpha)}\sqrt{4\bar{k} \bar{z}_o^2 - 4k \bar{z}_k \bar{z}_o + z_{(1-\alpha)}^2})}{2\bar{z}_o^2}.
\end{align*}
In both cases, if the number of studies needed to bring the combined $p$-value to the significance threshold $\alpha$ is rather low (Rosenthal suggests ${5k + 10}$ as a tentative threshold), one should be wary of potential publication bias. If the number is very high, one can be more confident that there is indeed an effect, but even in this latter case one cannot rule out publication bias. Rosenthal's file drawer estimator and the correction outlined above only yield worst-case estimates for the number of omitted studies assuming the null effect is true. If there is a real effect, one needs to resort to other methods to gauge the number of potentially omitted studies.\par
Table~\ref{tab:file_drawer} shows that the detection of publication bias not only fails if there is a real effect but also if the worst-case assumption holds, that is, if one assumes that all published studies consist exclusively of Type I errors. In the example simulations shown in Figure~\ref{fig:funnel_plot}, the file drawer method only detects publication bias if the true effect is zero and the body of published studies consists of all significant results plus $10\%$ of non-significant results.\par
\begin{table}[h!]
  \begin{center}
    \begin{tabular}{ >{\raggedright\let\\\tabularnewline}p{.4\textwidth} | >{\raggedleft\let\\\tabularnewline}p{.075\textwidth}| >{\raggedleft\let\\\tabularnewline}p{.075\textwidth} | >{\raggedleft\let\\\tabularnewline}p{.075\textwidth} | >{\raggedleft\let\\\tabularnewline}p{.075\textwidth} | >{\raggedright\let\\\tabularnewline}p{.1\textwidth}} 
    \hline
     & $k$\TBstrut & $\mu_1$ & $o$ & $o^*$ & bias detected?\\
    \hline
    full sample (no bias, A)\Tstrut & $200$ & $0$ & $-141$ & $-181$ & $o\phantom{^*}$: no $o^*$: no \\ 
    & $200$\Bstrut & $0.3$ & $8123$ & $884$ & $o\phantom{^*}$: no $o^*$: no\\
    \hline
    significant studies (B)\Tstrut& $13$ & $0$ & $320$ & $109$ & $o\phantom{^*}$: no $o^*$: no \\
    & $38$\Bstrut & $0.3$ & $2705$ & $457$ & $o\phantom{^*}$: no $o^*$: no \\
    \hline
    \Tstrut significant studies and $10\%$ of non-significant studies (C)& $32$ & $0$ & $99$ & $43$ & $o\phantom{^*}$: yes $o^*$: yes \\
    & $55$\Bstrut & $0.3$ & $3086$ & $494$ & $o\phantom{^*}$: no $o^*$: no\\
 \hline
\end{tabular}
    \caption[File drawer estimates to detect Publication bias.]{The number of potentially suppressed studies $o$ (uncorrected) and $o^*$ (corrected) calculated for the examples shown in Figure~\ref{fig:funnel_plot}. To test for the presence of publication bias, the number of studies in the file drawer was compared to the threshold value proposed by Rosenthal ($5k+10$).}
    \label{tab:file_drawer}
  \end{center}
\end{table}
In addition to the potential shortcomings explained above, it is important to point out that calculating the file drawer in the manner described above hinges on the assumption that the primary outcomes, the hypotheses as well as the testing procedure have been defined by the authors at the onset of the study. If they resorted to any form of `$p$-hacking' \citep{head_extent_2015} or `HARKing: Hypothesising After the Results are Known' \citep{kerr_harking_1998}, however, the expected Type I error would be much larger than indicated by the predefined significance threshold because it is rather easy to turn spurious results into significant findings by resorting to the strategies outlined above \citep{simmons_falsepositive_2011}.\par
Hence, the number of theoretically omitted studies that is needed to turn published results non-significant would be considerably lower because the source of bias is not primarily the omission of studies who show non-significant findings but rather the violation of fundamental principles of significance testing. 
%toadd: Maybe add a plot showing how Rosenthals estimatore overestimates the number of omitted studies
%to add: Orwin, R. G. (1983). A fail-safe N for effect size in meta-analysis. Journal

\subsection{How many significant studies should be expected?}
One approach to circumvent the shortcomings of the file drawer estimation in the presence of $p$-hacking and HARKing was developed by \citet{ioannidis_exploratory_2007}. It hinges on the idea that---given a real effect $\theta_j$---the probability of detecting this effect using a significance test equals the power $1-\beta_j$ of said test. If we furthermore assume that the effect is the same for all studies assessing the same question, the expected number of significant results is given by $$E = \sum_{i=1}^k (1-\beta_j).$$This expected number can then be compared against the actually observed number of studies reporting significant results $O$ by means of the $\chi^2$ test statistic $$A = [(O-E)^2/E + (O-E)^2/(k-E)] \sim \chi_1^2$$ 
and the corresponding decision function
\begin{align*}
    \delta(A) = \begin{cases} 1, & \text{if } A > q_{(\chi_1^2,1-\alpha)}; \\ 0, & \mbox{otherwise}.\end{cases}
\end{align*}
To calculate the power $\beta_j$ of each study, Ioannidis and Trikalinos suggested using the observed aggregate effect size $\hat{\theta}_k$, while admitting that this overestimates $E$, because `most biases that increase the proportion of `positive' results may also inflate the observed summary effect size' (p.~246). To alleviate this, they recommend considering a range of effect sizes derived from external evidence.\par
Table~\ref{tab:expected_significance} shows the results of this test for the simulated examples shown in Figure~\ref{fig:funnel_plot}. All instances of publication bias are correctly detected.\par
\begin{table}[h!]
  \begin{center}
    \begin{tabular}{ >{\raggedright\let\\\tabularnewline}p{.4\textwidth} | >{\raggedleft\let\\\tabularnewline}p{.075\textwidth}| >{\raggedleft\let\\\tabularnewline}p{.075\textwidth} | >{\raggedleft\let\\\tabularnewline}p{.075\textwidth} | >{\raggedleft\let\\\tabularnewline}p{.075\textwidth} | >{\raggedright\let\\\tabularnewline}p{.1\textwidth}} 
    \hline
     & $k$\TBstrut & $\mu_1$ & $\bar{x}_N$ & $A$ & bias detected?\\ 
    \hline
    full sample (no bias, A)\Tstrut & $200$ & $0$ & $-0.02$  & $1.70$  & no \\ 
    & $200$\Bstrut & $0.3$ & $0.27$  & $0.08$ & no\\
    \hline
    significant studies (B)\Tstrut& $13$ & $0$ & $0.69$ & $11.67$ & yes \\
    & $38$\Bstrut & $0.3$ & $0.61$ & $19.94$ & yes \\
    \hline
    significant studies and $10\%$ of\Tstrut & $32$ & $0$ & $0.15$ & $28.64$ & yes \\
    non-significant studies (C) & $55$\Bstrut & $0.3$ & $0.51$ & $8.21$ & yes\\
 \hline
\end{tabular}
    \caption[$\chi^2$-test to test for concordance between the expected and observed number of significant studies.]{Results of the $\chi^2$-test to test for concordance between the expected and observed number of significant studies. The test was conducted for the studies shown in Figure~\ref{fig:funnel_plot} and was performed at $\alpha = 0.05$. The global estimate $\bar{x}_N$ was calculated according to Eq.~\ref{eq:global_estimator_precision_weight} but with theoretical parameter values replaced by empirical estimates. The power of each study $j$ was calculated using Eq.~\ref{eq:power_calculation} with $\tau_1$ equal to the global estimate divided by the theoretical standard error of each study $j$.}
    \label{tab:expected_significance}
  \end{center}
\end{table}

Alternatively, Ioannidis and Trikalinos suggest to use a binomial probability test,  which is only advisable, however, if the power values $\beta_j$is roughly the same across studies $j$. If this is not the case, it is advisable to use the Poisson binomial test statistic instead with ${B_j \sim \text{Ber}( 1-\beta_j)}$ and therefore ${E  = \sum_{j=1}^k B_j \sim \text{Poisbin}(k, [(1-\beta_1),\dots,(1-\beta_k)])}$ so that the decision function would be 
\begin{align*}
    \delta(E) = \begin{cases} 1, & \text{if } E > q_{(1-\alpha)}; \\ 0, & \mbox{otherwise;}\end{cases}
\end{align*}
with $q_{(1-\alpha)}$ the $(1-\alpha)$-quantile of the Poisson binomial distribution outlined as above.\par

\subsection{Funnelling statistical evidence}
\label{subsec:funnel plot}
The so-called `funnel plot' is probably the best known and most widely adopted instrument to detect publication bias. Described as early as 1984 by \citet[p.~64--69]{light_summing_1984}, it has since become a standard method for meta-analysts to visually assess the extent of publication bias present in the published literature.\par
Funnel plots are usually constructed by plotting some measure of effect size against some measure of precision. Without publication bias, the study results should be distributed around the true effect size $\theta$, thereby creating an inverted funnel with the most precise studies at the top of the funnel and the rest of the studies symmetrically spreading out around the population mean towards the bottom of the graph (see Figure~\ref{fig:funnel_plot}). If publication bias is present, however, one would expect a `gap' in the funnel where those studies with little precision and non-significant findings were expected to be.\par
The correct choice of the axes often depends on the question at hand (see for example \citet[p.~81--89]{sterne_funnel_2005}). Often, a summary statistic of the effect size such as the mean or the log odds ratio is used for the abscissa whereas the values on the ordinate correspond to the standard error, the variance, their respective inverse, or simply the sample size \citep{sterne_funnel_2001}. Researchers have shown that the variety of possible choices for the axes can severely impact the robustness of the funnel plot as a diagnostic tool, since a different axis often leads to a different assessment of the presence or absence of publication bias \citep{tang_misleading_2000, lau_case_2006}. In addition, it was shown that mere visual inspection of funnel plots is, in general, not enough to reliably detect publication bias and might even be misleading, even for experienced systematic reviewers \citep{terrin_empirical_2005}. This can also be observed in Figure~\ref{fig:funnel_plot}: Even though both funnel plots on the last row are generated from a biased sample, the proclaimed funnel asymmetry is only observable if there are no non-significant studies published (B). Adding just $10\%$ of randomly selected non-significant studies makes the funnels appear much more symmetric. This shows that visual inspection of funnel plots can be rather misleading.\par
With `Egger regression' and the rank correlation test developed by Begg and Mazdumar, two non-visual tools to detect publication bias based on the rationale of the funnel plot exist but they both suffer from similar drawbacks as the funnel plot itself if the publication bias is not very pronounced or the number of studies is low \citep{sterne_regression_2005}. The following two sections contain a quick overview over both of them.

\myfig{ch3_fig1_funnel_plot}%% filename in figures folder
  {width=\textwidth,height=\textheight}%% maximum width/height, aspect ratio will be kept
  {Funnel plots of $200$ randomly simulated studies with sample sizes ${n \in [5,10,20,30,40,50,100]}$ and equal sampling probability for each $n$. The funnel plots are shown for ${H_1: \mu = 0}$ (left column) and ${H_1: \mu = 0.3}$ (right column) with ${H_0: \mu = 0}$ in both cases. Row A shows the complete sample, row B only significant studies and row C significant studies plut $10\%$ of non-significant studies. The solid black line on the vertical axis denotes the true population mean whereas the dotted black line denotes the significance threshold at different levels of precision. Individual data points in each simulated study were independently drawn from a $\mathcal{N}(\mu_1,4)$-distribution. To determine whether a given $\bar{x}_n$ was significant, $V_n$ was calculated as described in Eq.~\ref{eq:Vn_stud} and then compared to the $z_{(1-\alpha)}$-quantile with $\alpha = 0.05$.} %% caption
  {Funnel Plots in the presence and absence of publication bias.}%% optional (short) caption for table of figures
  {fig:funnel_plot}%% label
\clearpage
\subsection{Rank correlation between effect size and standard error}
\citet{begg_operating_1994} developed a simple non-parametric test for the detection of publication bias, calling it `a direct statistical analogue of the popular ``funnel-graph'' ' (p.~1088). They exploited the fact that in the presence of publication bias, `if negative studies are less likely to be published, the [funnel] graph will tend to be skewed, inducing a negative correlation in the graph, or, expressed differently, a positive correlation between estimates of effects and their variances' (p.~1088).\par
Let $X$ be a normally distributed random variable with unknown expectation $\mu$ and known variance $\sigma^2$. Suppose researchers estimate $\mu$ by the sample mean $\bar{\mu}_{n_j} = \hat{X}_{n_j} \sim \mathcal{N}(\mu, \sigma^2/n_j)$, where $n_j$ denotes the sample size of the $j$th study. The global estimate of the effect size across all $k$ studies (assuming fixed effects) can be calculated by averaging the sample mean $\bar{X}_{n_j}$ of each study $j$ weighted by the inverse of the sampling variance of that study:
\begin{align}
    \bar{X}_N = \frac{\sum_{j=1}^k \bar{X}_{n_j}w_j}{\sum_{j=1}^k w_j} \sim \mathcal{N}(\mu,\sigma^2/N) \label{eq:global_estimator_precision_weight}
\end{align}
where $N = \sum_{j=1}^k n_j$ and $w_j = n_j/\sigma^2$. With this, one can center each study estimate around the global estimate of the effect size
$$(\bar{X}_{n_j}-\bar{X}_N) \sim \mathcal{N}(0, v_j)$$
with $v_j = w_j - 1/\sum_{i=1}^k w_i$ denoting the sampling variance of the centred effect estimate $(\bar{X}_{n_j}-\bar{X}_N)$. Finally, one can stabilise the variance to $1$ by dividing the centred effect estimate by the square root of its sampling variance
\begin{align}
    Z_j = \frac{\bar{X}_{n_j}-\bar{X}_N}{\sqrt{v_j}} \sim \mathcal{N}(0, 1) \label{eq:standardised_effect_size}
\end{align}
which yields a test statistic following a standard normal distribution. Both  $Z_j$ and $v_j$ can then be used to calculate Kendall's $\tau$ \citep{kendall_new_1938} to test for correlation between the observed standardised effect sizes $z_j$ and their corresponding variances $v_j$:
$$\tau = \frac{\sum_{i=1}^{k-1}\sum_{j=i+1}^k \text{sgn}(z_i-z_j)\text{sgn}(v_i-v_j)}{k(k-1)/2} = \frac{c-d}{k(k-1)/2}$$
where $c$ denotes the number of concordant pairs—that is, those pairs of studies $i$ and $j$ for which $z_i$ and $v_i$ are both larger or both smaller than $z_j$ and $v_j$. Conversely, $d$ denotes the number of discordant pairs, that is, those pairs of studies $i$ and $j$ for which $z_i$ is larger than $z_j$ whereas $v_i$ is smaller than $v_j$ or vice versa. In other words, $\tau$ gives the ratio of concordant pairs minus discordant pairs divided by the total number of possible permutations of pairs $k(k-1)/2$. Under the null hypothesis of no correlation between effect size $z$ and sampling variance $v$ and with $k$ sufficiently large, the sampling distribution of $\tau$ is approximately normal with standard error $\sigma_{\tau} = \sqrt{\frac{2(2k+5)}{9k(k-1)}}$ \citep{kendall_new_1938}. Hence, we can construct the following statistic to test whether one should assume publication bias
$$Z_k = \frac{\tau}{\sigma_{\tau}} = \frac{c-d}{\sqrt{(2k+5)k(k-1)/18}} \sim \mathcal{N}(0,1).$$
The test statistic can then be passed to the following two-sided decision function
\begin{align*}
    \delta(Z_k) = \begin{cases} 1, & \text{if } |Z_k| > z_{(1-\alpha/2)}; \\ 0, & \mbox{otherwise}.\end{cases}
\end{align*}
with $z_{(1-\alpha/2)}$ denoting the $(1-\alpha/2)$-quantile of the standard normal distribution.\par
Going back to the simulated studies presented in Figure~\ref{fig:funnel_plot}, we can now calculate the test statistic for each simulation. As Table~\ref{tab:rank_correlation} shows, the test detects publication bias when it is also clearly recognizable visually but fails to do so when the bias is less obvious.\par
\begin{table}[h!]
  \begin{center}
    \begin{tabular}{ >{\raggedright\let\\\tabularnewline}p{.4\textwidth} | >{\raggedleft\let\\\tabularnewline}p{.075\textwidth}| >{\raggedleft\let\\\tabularnewline}p{.075\textwidth} | >{\raggedleft\let\\\tabularnewline}p{.075\textwidth} | >{\raggedleft\let\\\tabularnewline}p{.075\textwidth} | >{\raggedright\let\\\tabularnewline}p{.1\textwidth}} 
    \hline
     & $k$\TBstrut & $\mu_1$ & $\bar{x}_N$ & $|\tau/\sigma_{\tau}|$ & bias detected?\\ 
    \hline
    full sample (no bias, A)\Tstrut & $200$ & $0$ & $-0.02$  & $1.33$ & no \\ 
    & $200$\Bstrut & $0.3$ & $0.27$  & $0.51$ & no\\
    \hline
    significant studies (B)\Tstrut& $13$ & $0$ & $0.69$ & $2.81$ & yes \\
    & $38$\Bstrut & $0.3$ & $0.61$ & $5.70$ & yes \\
    \hline
    significant studies and $10\%$ of\Tstrut & $32$ & $0$ & $0.15$ & $0.36$ & no \\
    non-significant studies (C) & $55$\Bstrut & $0.3$ & $0.51$ & $1.60$ & no\\
 \hline
\end{tabular}
    \caption[Rank correlation test to detect publication bias.]{The rank correlation between effect size and variance calculated for the studies shown in Figure~\ref{fig:funnel_plot}. The significance test was performed at $\alpha = 0.05$ and the global estimate $\bar{x}_N$ was calculated according to Eq.~\ref{eq:global_estimator_precision_weight} but with theoretical parameter values replaced by empirical estimates.}
    \label{tab:rank_correlation}
  \end{center}
\end{table}
\subsection{Egger regression}
\citet{egger_bias_1997} used a regression-based approach to assess the correlation between the standardised effect size of a study ($z_j$, as described in Eq.~\ref{eq:standardised_effect_size}) and the corresponding precision defined as the inverse of the empirical standard error of the non-standardised effect size ($1/s_j$):
$$z_j \sim \beta_0 + \beta_1/s_j.$$
If there is no publication bias, the relationship between $z_j$ and $1/s_j$ should be completely determined by $\beta_1$, because standardised effect sizes should be low, on average, for studies with high standard errors (low precision) and high, on average, for studies with low standard errors (high precision). Hence, the regression line should pass through the origin, that is, intercept $\beta_0 = 0$. However, if there is publication bias present, we should expect higher standardised effect sizes for studies with low precision.\par
To test whether $\hat{\beta}_0$ differs significantly from $\beta_0 = 0$, the following $t$-statistic can be constructed:
$$T_k = \frac{\hat{\beta}_0-\beta_0}{s_{\hat{\beta}_0}} = \frac{\hat{\beta}_0}{s_{\hat{\beta}_0}} \sim t(\nu = k-2)$$
with $s_{\hat{\beta}_0}$ denoting the standard error of $\hat{\beta}_0$, $k$ denoting the total number of studies and $\nu = k-2$ the degrees of freedom of the $t$-distribution. The statistic can be passed to the following two-sided decision function: 
\begin{align*}
    \delta(T_k) = \begin{cases} 1, & \text{if } |T_k| > q_{\blb t_{k-2},1-\alpha/2\brb}; \\ 0, & \mbox{otherwise}.\end{cases}
\end{align*}
where $q_{\blb t_{k-2},1-\alpha/2 \brb}$ denotes the $(1-\alpha/2)$-quantile of the central $t(k-2)$-distribution. The Table~\ref{tab:Egger_regression} shows the results of the Egger regression test for the examples displayed in Figure~\ref{fig:funnel_plot}. As before, the test is able to detect publication bias that is visually recognisable, but often fails to detect publication bias if it is less pronounced.
\begin{table}[h!]
  \begin{center}
    \begin{tabular}{ >{\raggedright\let\\\tabularnewline}p{.4\textwidth} | >{\raggedleft\let\\\tabularnewline}p{.075\textwidth}| >{\raggedleft\let\\\tabularnewline}p{.075\textwidth} | >{\raggedleft\let\\\tabularnewline}p{.075\textwidth} | >{\raggedleft\let\\\tabularnewline}p{.075\textwidth} | >{\raggedright\let\\\tabularnewline}p{.1\textwidth}} 
        \hline
         & $k$\TBstrut & $\mu_1$ & $\bar{x}_N$ & $|t|$ & bias detected?\\ 
        \hline
        full sample (no bias, A)\Tstrut & $200$ & $0$ & $-0.02$  & $1.30$ & no \\ 
        & $200$\Bstrut & $0.3$ & $0.27$  & $0.95$ & no\\
        \hline
        significant studies (B)\Tstrut& $13$ & $0$ & $0.69$ & $3.77$ & yes \\
        & $38$\Bstrut & $0.3$ & $0.61$ & $8.54$ & yes \\
        \hline
        significant studies and $10\%$ of\Tstrut & $32$ & $0$ & $0.15$ & $1.02$ & no \\
        non-significant studies (C) & $55$\Bstrut & $0.3$ & $0.51$ & $4.05$ & yes\\
     \hline
    \end{tabular}
    \caption[Egger regression test to detect publication bias.]{The Egger regression test to detect correlation between effect size and variance calculated for the studies shown in Figure~\ref{fig:funnel_plot}. The significance test was performed at $\alpha = 0.05$ and the global estimate $\bar{x}_N$ was calculated according to Eq.~\ref{eq:global_estimator_precision_weight} but with theoretical parameter values replaced by empirical estimates.}
    \label{tab:Egger_regression}
  \end{center}
\end{table}
%toadd: test statistic and decision function for Egger regression
%toadd: Peters 2006, comparison of two methods to detect publication bias
%toadd: Macaskill: a comparison of methods to detect publication bias in meta-analysis
%toadd: Moreno: assessment of regression-based methods to adjust for publication bias
%toadd: selection model described by Begg! Could be useful for my selection simulations

\subsection{The calliper test: Pinching significance thresholds}
A calliper is a mechanical instrument used to precisely measure the diameter of (small) objects by clamping them in between the two jaws of callipers. Inspired by this measuring device, \citet{gerber_can_2006, gerber_publication_2008} developed the so-called `calliper test', which serves to detect distributional discontinuities of the $p$-value around significance levels. Without significance-driven publication bias, the number of publications reporting barely significant findings should be roughly the same as the number of publications reporting $p$-values just above the $\alpha$-threshold. \par

Let us assume that there is a unknown effect size $\mu$ and known variance $\sigma^2$ so that ${X \sim \mathcal{N}(\mu,\sigma^2)}$. Then, ${\bar{X} = \sum_{i_1}^n \sim \mathcal{N}(\mu, \frac{\sigma^2}{n})}$ and ${Z = \sqrt{n}\frac{\bar{X}}{\sigma} \sim \mathcal{N}(\sqrt{n}\frac{\mu}{\sigma}, 1)}$. If $X$ is not normally distributed or the variance $\sigma^2$ is unknown, these distributional properties hold asymptotically.\par

Let $\Phi(\cdot)$ be the cumulative distribution function of the standard normal distribution. For the limits $c < c' < c''$, the conditional probability that an observed $Z$-scores lies in the interval $[c,c']$, given that it is drawn from the interval $[c, c'']$, corresponds to 
\begin{align}
    \Pr[z \in [c,c'] \mid z \in [c,c'']] = \frac{\Phi(c')-\Phi(c)}{\Phi(c'')-\Phi(c)}. \label{eq:calliper1}
\end{align}
Since $\Phi$ is continuous, it holds that $\Phi(z+e) - \Phi(z) = e\phi(z) + \epsilon$ where $\epsilon$ is an approximation error which goes to zero as $e$ approaches zero. Therefore, ratio \ref{eq:calliper1} can be approximated by
\begin{align}
    \Pr[z \in [c,c'] \mid z \in [c,c'']] \simeq \frac{(c'-c)\phi(c)}{(c''-c)\phi(c)} = \frac{c'-c
    }{c''-c} \label{eq:calliper2}
\end{align}
for small intervals $[c,c'']$ (that is, small $e$) and small $\phi(\cdot)$ within the interval. If we choose $c, c'$ and $c''$ so that ${c'- c = c''-c'}$, then 
\begin{align}
    p = \Pr(z \in [c,c'] \mid z \in [c,c'']) = 0.5 \label{eq:calliper3}
\end{align}
that is, the conditional probability of a $Z$-score being an element of either the upper or lower half of the interval is fifty per cent. Equations \ref{eq:calliper2} and \ref{eq:calliper3} can now be used to construct an exact binomial test to test for publication bias.\par

Let $k$ be the total number of published studies and let $k'$ be the number of studies that fall into the interval $[c,c'']$. If the $Z$-scores reported in the published studies represent $k$ independent draws from $Z \sim \mathcal{N}(\sqrt{n}\frac{\mu}{\sigma},1)$, then the number of studies $k'' \leq k'$ falling into the upper (or lower) half of $[c, c'']$ is a realisation of a binomially distributed variable $K'' \sim \text{Bin}(k', p = 0.5)$. \par
To test for publication bias of results from one-sided superiority tests, one can set ${c' = z_{(1-\alpha)}}$ and ${[c,c''] = [c'-e, c+e]}$ with 
$z_{(1-\alpha)}$ the $(1-\alpha)$-quantile of the standard normal distribution and $e$ small. The test hypotheses are
\begin{align*}
    H_0 &: p = 0.5 \\
    H_1 &: p \neq 0.5.
\end{align*}
with test statistic $k''$ and decision function 
\begin{align*}
    \delta(k'') = \begin{cases} 1, & \text{if } k'' > q_{(1-\alpha/2)}; \\
    1, & \text{if } k'' < -q_{(1-\alpha/2)}, \\
    0, & \mbox{otherwise};\end{cases}
\end{align*}
where $q_{(1-\alpha/2)}$ denotes the ${(1-\alpha/2)}$-quantile of the binomial distribution ${\text{Bin}(k', p = 0.5)}$.\par
Table~\ref{tab:calliper_test} shows the results of the calliper test for the simulated studies depicted in Figure~\ref{fig:funnel_plot}. Due to the small number of studies lying in the test window $[c,c'']$, the test has low power in these examples and publication bias is only detected in one case.
\begin{table}[h!]
  \begin{center}
    \begin{tabular}{ >{\raggedright\let\\\tabularnewline}p{.4\textwidth} | >{\raggedleft\let\\\tabularnewline}p{.075\textwidth}| >{\raggedleft\let\\\tabularnewline}p{.075\textwidth} | >{\raggedleft\let\\\tabularnewline}p{.075\textwidth} | >{\raggedleft\let\\\tabularnewline}p{.075\textwidth} | >{\raggedright\let\\\tabularnewline}p{.1\textwidth}} 
        \hline
         & $k$\TBstrut & $\mu_1$ & $k'$ & $k''$ & bias detected?\\ 
        \hline
        full sample (no bias, A)\Tstrut & $200$ & $0$ & $4$ & $3$ & no \\ 
        & $200$\Bstrut & $0.3$ & $15$ & $6$ & no\\
        \hline
        significant studies (B)\Tstrut& $13$ & $0$ & $3$ & $3$ & no \\
        & $38$\Bstrut & $0.3$ & $6$ & $6$ & yes \\
        \hline
        significant studies and $10\%$ of\Tstrut & $32$ & $0$ & $3$ & $3$ & no \\
        non-significant studies (C) & $55$\Bstrut & $0.3$ & $7$ & $6$ & no\\
     \hline
    \end{tabular}
    \caption[The calliper test to detect publication bias.]{The calliper test to detect publication bias for the examples shown in Figure~\ref{fig:funnel_plot}. The window size $e$ was set to $0.2$ and the exact binomial test was performed at $\alpha = 0.05$.}
    \label{tab:calliper_test}
  \end{center}
\end{table}
As opposed to the file drawer calculations proposed by \citet{rosenthal_file_1979}, the calliper test can theoretically also detect publication bias due to $p$-hacking and HARKing because these manipulations directly influence the conditional probabilities of finding studies in the upper or lower part of a small interval around the significance threshold.\par
It should be noted, however, that the test is only approximately accurate if the sampling distribution of $Z$ has a mean value close to the critical value, that is, the $(1-\alpha)$-quantile of the standard normal distribution. If the sampling distribution is not symmetric around the critical value, $\Pr(z \in [c,c+e] \mid z \in [c-e,c+e])$ does not equal 0.5 anymore. However, for very large deviations of the mean value from the critical value, the overall probability of any values falling into $[c-e,c+e]$ is negligible. Coincidentally, in those cases a test for significance-driven publication bias is rendered moot, since all or most of the results are expected to be significant because of the presence of a strong real effect. %toadd: maybe add quick calculations

\subsection{The \texorpdfstring{$p$}{p}-curve}
\label{subsec:p-curve}
As already mentioned in Section \ref{subsec:file drawer}, not all methods that try to detect significance-driven publication bias are robust against $p$-hacking and HARKing. The $p$-curve test was specifically designed by \citet{simonsohn_pcurve_detection_2014} to provide a robust method to detect publication bias in the presence of such manipulations. The rationale of the test is based on the observation that the distribution of $p$-values is uniform if the null hypothesis is true\footnote{This only holds for simple null hypothesis and composite null hypotheses evaluated at the most unfavourable null parameter value.} but should be right-skewed if the alternative holds (see Section \ref{subsec:p-value} for more details).\par
Simonsohn \textit{et al.} proposed using these properties by aggregating all significant $p$-values pertinent to a specific set of hypotheses in order to assess whether a body of published studies contained any evidential value or not. Let 
$${\gamma = \frac{1/n \sum_{j=1}^l (p_j-\bar{p}_l)^3}{[1/(n-1)\sum_{j=1}^l (p_j-\bar{p}_l)^2]^{3/2}}}$$ 
be the sample skewness of $p$ below the significance threshold. We can then distinguish between the following four cases:
\begin{align*}
    \text{1. } &H_0 \text{ true and $p$-hacking absent:} &&P \sim \text{Unif}(0,1) &&\text{ (no skew)};\\
    \text{2. } &H_0 \text{ true and $p$-hacking present:} &&\gamma < 0 &&\text{ (left skew)};\\
    \text{3. } &H_0 \text{ false and $p$-hacking absent:} &&\gamma > 0 &&\text{ (right skew)};\\
    \text{4. } &H_0 \text{ false and $p$-hacking present:} &&\gamma < 0 &&\text{ (left skew) or }\\ 
    & &&\gamma > 0 &&\text{ (right skew).}
\end{align*}
Cases 1 to 3 can relatively easy be distinguished, for example by visual inspection of the empirical distribution of $p$-values or by calculating the sample skewness and testing whether it significantly deviates from zero skew expected under $P \sim \text{Unif}(0,1)$. Distinguishing Case 4 from Cases 2 and 3 is more subtle, however. Simnsohn \textit{et al.} did not offer a method to do so, but instead suggested to reframe the problem as a test for the presence or absence of evidential value.\par
To test whether a certain set of significant findings $l$ contains real evidence against the null, one first calculates for each observed $p$-value the probability of observing a $p$-value that is at least as extreme if the null were true and given that the observed $p$-value is significant, that is, ${\Pr(P \leq p \mid p < \alpha)}$. Simonsohn \textit{et al.} call this the `$pp$-value'---the $p$-value of the $p$-value. They then combine the $pp$-values from all $k$ significant findings to construct the test statistic for Fisher's combined probability test statistic $$pp_{\text{comb}} = -2\sum_{j=1}^l \text{ln}(pp_j)$$
which follows a $\chi_{2s}^2$-distribution if the $pp$-values are drawn from a $\text{Unif}(0,1)$-distribution. Hence, the test statistic can be passed to the following decision function to test for significant skewness:
\begin{align*}
    \delta(pp_{\text{comb}}) = \begin{cases} 1, & \text{if } pp_{\text{comb}} > q_{(\chi_{2s}^2,1-\alpha)}; \\ 
    0, & \mbox{otherwise}.\end{cases}
\end{align*}
If the distribution $p$-values are significantly left skewed, that is, ${\gamma < 0}$ and ${\delta(pp_{\text{comb}}) = 1}$, the analysed results were probably $p$-hacked. If the distribution of $p$-values is significantly right-skewed, (${\gamma > 0}$ and ${\delta = 1}$), $p$-hacking might still have occurred, but one can at least conclude that the studies reporting significant findings did indeed contain some evidential value against the null. However, if Fisher's combined probability test turns out to be non-significant, this may indicate either that there are not enough findings to draw any conclusion or that the findings did not contain evidential value, that is, that they originated from $p$-hacking.\par
To distinguish between these two cases, Simonsohn \textit{et al.} suggested conducting a second test, but this time against a null hypothesis that assumes a very small effect instead of no effect. Specifically, they proposed to compare the observed $p$-curve to a $p$-curve that would be expected for studies with a low power of $0.33$. This curve can be constructed recalculating the $pp$-values under the assumption that the underlying $p$-values originated from studies with a small real effect and corresponding power of $0.33$.\par
Let us follow through with the example inspired by Simonsohn \textit{et al.} and assume that the original $p$-values were calculated using Student's $t$-statistic (see \ref{subsec:student's t-statistic} for more details). Under the null hypothesis $H_0$, the $t$-statistic follows a central $t(\nu)$-distribution, with $\nu = n-1$ degrees of freedom and $n$ denoting the sample size used to calculate the statistic. The corresponding $p$-value is given by
$$p_{H_0} = \Pr(T>t_n \mid H_0) = 1-F_T(t_n\mid H_0)$$
with $F_{T}(\cdot \mid H_0)$ the cumulative distribution function of the central $t(\nu)$-distribution.\par
Under the alternative hypothesis $H_1$, the $t$-statistic follows a non-central $t(\lambda,\nu)$-distribution, with non-centrality parameter ${\lambda = \sqrt{n}(\mu-\mu_0)/\sigma}$ and $\nu$ and $n$ same as above.\par
Given a $t$-statistic $t_n$, sample size $n$ and the $(1-\alpha)$-quantile of the central $t(\nu)$-distribution $q_{(1-\alpha)}$, we can find the non-centrality parameter $\lambda$ corresponding to a power of $1-\beta$ as follows:
\begin{alignat*}{2}
    \phantom{\Longleftrightarrow}\quad &\Pr(T > q_{(1-\alpha)} \mid H_1) &&= 1-\beta\\
    \Longleftrightarrow\quad &1-F_{T}(q_{(1-\alpha)} \mid H_1) &&= 1-\beta\\
    \Longleftrightarrow\quad &F^{-1}_{T}(1-\beta\mid H_1) &&= q_{(1-\alpha)}
\end{alignat*}
with $F_{T}(\cdot \mid H_1)$ and $F_{T}^{-1}(\cdot \mid H_1)$ the cumulative distribution function and the quantile function of $T$ under the alternative hypothesis. In other words, we need to find the non-central $t(\lambda,\nu)$-distribution whose $0.67$-quantile is equal to the $q_{(1-\beta)}$-quantile of the central $t(\nu)$-distribution. Having done that, we can calculate the $p$-value of $t_n$ under the alternative which amounts to 
$$p_{H_1} = \Pr(T>t_n \mid H_1) = 1-F_T(t_n\mid H_1).$$ 
The corresponding $pp$-value is then
\begin{alignat*}{1}
    \Pr(P<p_{H_1} \mid p_{H_0} < \alpha, H_1) &= \frac{\Pr(P<p_{H_1}, p_{H_0} < \alpha\mid H_1)}{\Pr(p_{H_0}<\alpha)\mid H_1)}\\ 
    &= \frac{\Pr(P<p_{H_1}, p_{H_1} < 1-\beta\mid H_1)}{\Pr(p_{H_1}<1-\beta \mid H_1)}\\ 
    &= \frac{\Pr(P<p_{H_1}\mid H_1) - (\Pr(p_{H_1} \geq 1-\beta \mid H_1))}{\Pr(p_{H_1}<1-\beta \mid H_1)}\\
    &= \frac{p_{H_1}-\beta}{1-\beta}.
\end{alignat*}
By calculating these alternative $pp$-values for each observation, one can construct a reference $p$-curve for assumed power $1-\beta$ that can be tested against the null hypothesis of uniform distribution of the $p$-values. If the reference $p$-curve is significantly right-skewed and thus more right skewed than the observed $p$-curve, one can conclude that the analysed studies contain less evidential value than the same amount of studies with power $1-\beta$. If one picks $1-\beta$ so that the studies would be clearly underpowered (such as $1-\beta = 0.33$), this would mean that evidential value is rather low.\par
However, if the observed $p$-curve is not significantly less right-skewed than the newly constructed reference curve, one cannot make a judgement about the presence or absence of any evidential value based on the $p$-curve.\par
As \citet{simonsohn_pcurve_detection_2014} show, their $p$-curve method has high power to detect that a set of studies has evidential value or lack thereof, even if the individual studies have rather low power or are even slightly $p$-hacked, respectively. However, an important prerequisite of these results is the judicious definition and use of selection criteria with which to select $p$-values from individual studies. For example, the $p$-curve analysis does only work if the analysed $p$-values are independent from each other, if they have a uniform distribution under the null and if they are linked to the same hypothesis of interest.\par
\begin{table}[h!]
  \begin{center}
    \begin{tabular}{ >{\raggedright\let\\\tabularnewline}p{.4\textwidth} | >{\raggedleft\let\\\tabularnewline}p{.075\textwidth}| >{\raggedleft\let\\\tabularnewline}p{.075\textwidth} | >{\raggedleft\let\\\tabularnewline}p{.075\textwidth} | >{\raggedleft\let\\\tabularnewline}p{.09\textwidth} | >{\raggedright\let\\\tabularnewline}p{.1\textwidth}} 
        \hline
         & $k$\TBstrut & $\mu_1$ & $\gamma$ & $pp_{\text{comb}}$ & bias detected?\\ 
        \hline
        full sample (no bias, A)\Tstrut & $200$ & $0$ & $0.55$ & $44.94$ & no \\ 
        & $200$\Bstrut & $0.3$ & $0.45$ & $116.67$ & no\\
        \hline
        significant studies (B)\Tstrut& $13$ & $0$ & $0.55$ & $44.94$ & no \\
        & $38$\Bstrut & $0.3$ & $0.45$ & $116.67$ & no \\
        \hline
        significant studies and $10\%$ of\Tstrut & $32$ & $0$ & $0.55$ & $44.94$ & no \\
        non-significant studies (C) & $55$\Bstrut & $0.3$ & $0.45$ & $116.67$ & no\\
     \hline
    \end{tabular}
    \caption[The $p$-curve test to check for uniformity of significant $p$-values under the null hypothesis.]{Results of the $p$-curve test to check for uniformity of significant $p$-values under the null hypothesis. The test was conducted for the studies shown in Figure~\ref{fig:funnel_plot} at an $\alpha$-level of $0.05$.}
    \label{tab:p_curve}
  \end{center}
\end{table}
In addition, if $p$-hacking and HARKing are completely absent and publication bias originates exclusively from an increased publication probability for significant findings, the $p$-curve fails to detect any bias. This happens because the $p$-curve method only looks at $p$-values below the significance threshold whose distribution remains unchanged in the absence of $p$-hacking and HARKing regardless of how many non-significant findings are suppressed. The $p$-curve tests conducted for the simulated studies shown in Figure~\ref{fig:funnel_plot} confirm this. As can be seen in Table~\ref{tab:p_curve}, the tests yields exactly the same result in all three scenarios.\par
%toadd: p-curve is related to previous work that has examined the distribution of p values (or their corresponding t or Z scores) reported across large numbers of articles (Card & Krueger, 1995; Gadbury & Allison, 2012; Gerber & Malhotra, 2008a, 2008b; Masicampo & Lalande, 2012; Ridley, Kolm, Freckelton, & Gage, 2007)

%First, when true effect sizes differ across studies, as they inevitably do, the funnel plot and the excessive significance approaches risk falsely concluding publication bias is present when in fact it is not (Lau, Ioannidis, Terrin, Schmid, & Olkin, 2006; Peters, Sutton, Jones, Abrams, & Rushton, 2007; Tang & Liu, 2000). 

\section{Correcting biased estimates}
\label{sec:correct_publication_bias}
The detection of publication bias is an important endeavour in itself, lest biased meta-analyses are used as a basis for future research or even policy decision. This is especially important in the case of the most extreme form of publication bias, in which all published findings about a specific questions are simply a false positive. However, having detected both the presence of a real effect as well as publication bias, it would be desirable to have tools at hand with which to correct biased effect size estimates. The following sections present a selection of methods to do so. 
%TOADD
%TOADD: ecause overestimated effect sizes are more likely to be significant than are underestimated ones, the published record systematically overestimates effect sizes (Hedges, 1984; Ioannidis, 2008; Lane \& Dunlap, 1978). (comes from simonshon 2014 - correction)
\subsection{Publication probabilities and truncated distributions}
\label{subsec:pub_prob and trunc_dist}
In the absence of publication bias---and assuming no other major source of bias---the distribution of published findings is given by the sampling distribution of the corresponding summary statistic. Let ${X_i \sim \mathcal{N}(\mu,\sigma^2)}$ be a random variable from which data points are drawn and let the corresponding summary statistic be ${S_{n_j} = \bar{X}_{n_j} = \frac{1}{n_j} \sum_{i=1}^{n_j} X_i \sim \mathcal{N}(\mu, \sigma^2/n_j)}$ for each study $j$.\par
If we calculate the mean of ${\bar{X}_{n_j}}$ weighted by the inverse of the standard error we get the global effect size estimate 
\begin{align*}
    \bar{X}_N = \frac{\sum_{j=1}^k \bar{X}_{n_j}w_j}{\sum_{j=1}^k w_j}
\end{align*}
which was already outlined in Equation~\ref{eq:global_estimator_precision_weight}. ${N = \sum_{j=1}^k n_j}$ is again the total number of individual samples summed over all studies and $w_j$ is the squared inverse of the theoretical ($w_j = n/\sigma_j^2$) or estimated ($w_j = n/s_j^2$) standard error of each estimate $j$.\par
In the presence of publication, however, the global estimate is biased upwards. The following sections introduce methods to correct this, most of which rely on the following definitions.\
\begin{description}[leftmargin=!,labelwidth=\widthof{\bfseries $\boldsymbol{E_4}$}]
    \item [\textbf{Publication rate:}] Let $X_1,\dots,X_n$ be a set of i.i.d random variables with expectation $\mu$. I define the publication probability of a finding based on a test statistic $S_n(X_1,\dots,X_n)$ as
\begin{align}
    {\ppr(S_n, \pi) = \pi + (1-\pi)\delta(S_n)} \label{eq:pub_prob}
\end{align}
where $\delta(\cdot)$ is a decision function taking $S_n$ as input and returning $1$ if $S_n$ lies beyond the predefined significance threshold and $0$ otherwise. Hence, if a finding is significant, its publication probability is assumed to be equal to $1$, if it is not, it is assumed to be equal to $\pi$. For the methods outlined below, $\pi$ is assumed to be a constant, but it could easily be replaced by function $\pi(\cdot)$, dependent on effect size, sample size, or other study properties.\\
\item [\textbf{Expected publication probability given $\boldsymbol{n}$:}] The expectation of the publication probability $\text{ppr}$ is given by
\begin{align}
    \E[\ppr(S_n, \pi)] = \int_0^1 p f_{\ppr}(p)dp \label{eq:expected_pub_prob}
\end{align}
with $f_{\ppr}(\cdot)$ denoting the probability density function of $\ppr$. If we assume $\pi$ to be constant, this simplifies to
\begin{align}
    \E[\ppr(S_n, \pi)] = \pi \Pr(\ppr = \pi) + \Pr(\ppr = 1) \label{eq:expected_pub_prob_constant}
\end{align}
\item [\textbf{Truncated probability density function of $\boldsymbol{S_n}$:}] Let $f_{S_n}(\cdot)$ be the probability density function of $S_n$. If publication bias is present, the truncated probability density function $ f_{S_n}^{*}(\cdot)$ is given by
\begin{align}
    f_{S_n}^{*}(s_n) = \frac{\ppr(s_n, \pi)}{\E[\ppr(S_n, \pi)]}f_{S_n}(s_n). \label{eq:truncated_prob_density}
\end{align}
\item [\textbf{Truncated likelihood of $\boldsymbol{\mu}$:}] The likelihood of $\mu$ given the observed test statistics $s_{n_1},\dots,s_{n_k}$ is defined as ${\mathcal{L}(\mu \mid s_{n_1},\dots,s_{n_k}) = \prod_{j=1}^k f_{S_{n_j},\mu}(
s_{n_j} \mid \mu)}$ with $f(s_{n_j} \mid \mu)$ denoting the probability density function of $S_{n_j}$ given $\mu$. In the presence of publication bias, I define the truncated likelihood of $\mu$ as
    \begin{align}
        \mathcal{L}^{*}(\mu \mid s_{n_1},\dots,s_{n_k}) &= \prod_{j=1}^k\frac{\ppr(s_{n_j}, \pi)}{\E[\ppr(S_{n_j}, \pi)\mid \mu]}f_{S_{n_j}}(s_{n_j})\\ \nonumber
        &=  \frac{\mathcal{L}(\mu \mid s_{n_1},\dots,s_{n_k})}{\E[\ppr(S_{n_j},\pi)\mid \mu]}\prod_{j=1}^k \ppr(S_{n_j},\pi) \label{eq:truncated_likelihood}
    \end{align}
\end{description}
The measures described above are taken from \citet{andrews_identification_2017}, but I use them for bias corrections methods that are different from the ones proposed by the two authors.%toadd: everything above given sample size n

\subsection{Reweighting estimates by publication probabilities}
Let ${S_{n_j} = \bar{X}_{n_j} = \frac{1}{n_j} \sum_{i=1}^{n_j} X_i \sim \mathcal{N}(\mu, \sigma^2/n_j)}$ be the summary statistic reported by a study $j$. To estimate the true effect size $\mu$ in the absence of publication bias, one can simply apply Eq.~\ref{eq:global_estimator_precision_weight}. However, if publication bias is present, this estimator is biased upwards.\par 
Let ${\ppr_j = \ppr(S_{n_j},\pi_j)}$ be the publication probability of a finding $S_{n_j}$ as stated in Section~\ref{subsec:pub_prob and trunc_dist}, Eq.~\ref{eq:pub_prob}, and assume that $\pi_j \neq 1$ for at least some non-significant findings. Then, $${\bar{X}_j^{*} = \bar{X}_j \ppr_j \sim \mathcal{N}(\mu\ppr_j, \sigma^2\frac{\ppr_j^2}{n_j}})$$ and the global estimator given in Eq.~\ref{eq:global_estimator_precision_weight} turns into by
\begin{align}
    \bar{X}_\text{N} = \frac{\sum_{j=1}^k \bar{X}_{n_j}^{*}w_j}{\sum_{j=1}^k w_j}\sim \mathcal{N}(\mu \frac{\sum_{j=1}^k n_j\ppr_j w_j}{\sum_{j=1}^k w_j},\sigma^2\frac{\sum_{j=1}^k n_j(\ppr_j w_j)^2}{(\sum_{j=1}^k w_j)^2})
\end{align} 
with ${w_j = n_j/\sigma^2}$. To correct the bias, on can simply reweight each finding $\bar{X}_j^{*}$ published under publication bias by the inverse of its publication probability, that is, $\bar{X}_j = \bar{X}_j^{*}/\ppr_j$ and then calculate the unbiased global estimate using Eq.~\ref{eq:global_estimator_precision_weight} which yields the following unbiased estimator:
\begin{align}
    \bar{X}_\text{N}^{*} = \frac{\sum_{j=1}^k \bar{X}_{n_j}^{*}w_j/\ppr_j}{\sum_{j=1}^k w_j/\ppr_j}\sim \mathcal{N}(\mu,\sigma^2/\sum_{j=1}^k w_j/\ppr_j) \label{eq:global_estimator_corrected}
\end{align} 
This method---inspired by the finite sample estimator developed by \citet{hansen_theory_1943}---performs well when there is a real but biased effect but underestimates the global effect size the assumed or estimated publication probability is smaller than the real one. For biased null effects the estimator also tends to underestimate the true effect size if In scenarios in which there are only significant findings observed, it completely fails (see Table~\ref{tab:hansen_hurvitz} for examples). In addition, the method is only applicable if the publication probability for a given study is known or can be estimated.\par
\begin{table}[h!]
  \begin{center}
    \begin{tabular}{ >{\raggedright\let\\\tabularnewline}p{.4\textwidth} | >{\raggedleft\let\\\tabularnewline}p{.1\textwidth}| >{\raggedleft\let\\\tabularnewline}p{.1\textwidth} | >{\raggedleft\let\\\tabularnewline}p{.1\textwidth} | >{\raggedleft\let\\\tabularnewline}p{.1\textwidth}} 
        \hline
         & $k$\TBstrut & $\mu_1$ & $\bar{x}_N$ & $\bar{x}_N^{*}$ \\ 
        \hline
        full sample (no bias, A)\Tstrut & $200$ & $0$ & $-0.02$  & $-0.07$ \\ 
        & $200$\Bstrut & $0.3$ & $0.27$  & $0.15$\\
        \hline
        significant studies (B)\Tstrut& $13$ & $0$ & $0.69$ & $0.69$ \\
        & $38$\Bstrut & $0.3$ & $0.61$ & $0.61$ \\
        \hline
        significant studies and $10\%$ of\Tstrut & $32$ & $0$ & $0.15$ & $-0.17$ \\
        non-significant studies (C) & $55$\Bstrut & $0.3$ & $0.51$ & $0.27$ \\
     \hline
    \end{tabular}
    \caption[Bias correction by reweighting with publication probabilities.]{The results of the bias correction based on reweighting each result by the inverse of its respective publication probability. The correction was applied to the studies shown in Figure~\ref{fig:funnel_plot}, the global estimate $\bar{x}_N$ was calculated according to Eq.~\ref{eq:global_estimator_precision_weight} and the corrected estimate $\bar{x}_N^{*}$ was calculated according to Eq.~\ref{eq:global_estimator_corrected}---in both cases with theoretical parameter values replaced by empirical estimates.}
    \label{tab:hansen_hurvitz}
  \end{center}
\end{table}

\subsection{Trim-and-fill: Closing gaps in funnel plots}
The `trim-and-fill' method developed by \citet{duval_trim_2000, duval_nonparametric_2000} leverages on a similar rationale as the funnel plot introduced in Section \ref{subsec:funnel plot}. It assumes that publication bias suppresses those findings which are most `negative' in the sense of `pointing in the opposite direction of significant findings'. As a result, we would expect a considerable gap in the lower left part of the funnel plot, where the those studies with low precision and negative effect sizes are located, and---conversely---a global effect size estimate that is biased upwards.\par %to add: figure of funnel plot showing most extreme findings missing
To correct for this, Duval and Tweedie propose an expectation-maximisation algorithm based on trimming the most `positive' findings, calculating a corrected global effect size estimate, again trimming the most positive findings with regard to the corrected estimate, re-correcting said estimate and so on, until convergence is achieved. Then, the most positive findings with regard to the converged global effect size estimate are mirrored around said estimate and a final global effect size estimate is calculated. In detail, the algorithm consists of the following stages (adapted from \citet{duval_trim_2005}): 
\begin{enumerate}
    \item \label{itm:transform} Transform the individual findings $\hat{\theta}_{1},\dots,\hat{\theta}_{k}$ into standard normally distributed test statistics ${V_{n_1}, \dots, V_{n_k}}$ using a variance stabilising transformation.
    \item \label{itm:global_estimate_init} Set $i=1$, $k_0^{(0)}=0$ and calculate a global estimate $V_N^{(i)}$ of the individual test statistics $V_{n_j}$.
    \item \label{itm:center} Centre the findings around $V_N^{(i)}$, that is, $V_{n_j}^{*} = V_{n_j} - V_N^{(i)}$.
    \item Rank all centred estimates according to their absolute value from smallest to largest and assign each rank the sign of its corresponding $V_{n_j}^{*}$. This yields signed ranks $\text{sr}_{V_{n_j}^{*}} = \text{sgn}(V_{n_j}^{*})\cdot\text{rank}_{|V_{n_1}^{*}|,\dots,|V_{n_k}^{*}|}(|V_{n_j}^{*}|)$, with $\text{rank}_{|V_{n_1}^{*}|,\dots,|V_{n_k}^{*}|}(\cdot)$ denoting the rank function with regard to absolute values of all centred estimates.
    \item Sum all ranks with positive sign to obtain the rank sum $$S_{\text{rank}} = \sum_{j=1}^k  \text{sr}_{V_{n_j}^{*}}\cdot \mathbb{1}(\text{sr}_{V_{n_j}^{*}} > 0).$$
    \item Estimate the number of potentially omitted studies $$k_0^{(i)} = \left\lceil\frac{4 S_{\text{rank}}-n(n+1)}{2n-1}\right\rceil.$$
    \item If ${k_0^{(i)} = k_0^{(i-1)}}$ go to step \ref{itm:fill}. Otherwise, continue with step \ref{itm:trim}. 
    \item \label{itm:trim} Trim off the $k_0^{(i)}$ most positive findings $V_{n_j}$ and recalculate the global estimate of the trimmed sample $V_N^{(i+1)}$.
    \item Set $i = i+1$ and restart from step \ref{itm:center} using all original findings (including the ones that were trimmed in step \ref{itm:trim}).
    \item \label{itm:fill} Take the $k_0^{(i)}$ most positive findings $V_{n_j}$, `mirror' them around $V_N^{(i)}$ and add them to the data set along with the corresponding standard error.
    \item Transform all $V_{n_j}$ back into effect size estimates $\hat{\theta}_j$ (including the newly filled data) and calculate a global effect size estimate $\hat{\theta}_{N}$.
\end{enumerate}
%to add: R implementation of code
%Let us assume that ${\hat{\theta}_j = \bar{x}_{n_j} = \frac{1}{n_j}\sum_{i=1}^{n_j} x_i}$ with ${X_i \sim \mathcal{N}(\mu,\sigma^2)}$ and thereby ${\bar{X}_{n_j} \sim \mathcal{N}(\mu, \sigma^2/n_j)}$. In \textbf{\textsf{R}}, the algorithm can be implemented as follows: 
%\begin{lstlisting}[language=R]
%test
%\end{lstlisting}
Since the trim-and-fill method hinges on the assumption that only the most extreme negative findings are omitted in the presence of publication bias, the correction will fail if the publication probability of a finding $\hat{\theta}_i$ does not decay with increasing negative distance from the significance threshold. If---for example---the publication probability of a finding is given by ${\ppr_j = \ppr(V_{n_j}, \pi_j)}$ as described in Section~\ref{subsec:pub_prob and trunc_dist}, Eq.~\ref{eq:pub_prob}, with ${V_{n_j} = V(\hat{\theta}_j)}$ a variance stabilised and normally distributed test statistic based on $\hat{\theta}_j$ and $\pi_j$ either constant across all $j$ or a function dependent on some study characteristic of interest, then the publication probability will be equal to $1$ above the significance threshold and $\pi_j$ below. In such cases, the trim-and-fill estimator will overestimate the true effect size as can be seen in Table~\ref{tab:trim_and_fill}.\par
\begin{table}[h!]
  \begin{center}
    \begin{tabular}{ >{\raggedright\let\\\tabularnewline}p{.4\textwidth} | >{\raggedleft\let\\\tabularnewline}p{.075\textwidth}| >{\raggedleft\let\\\tabularnewline}p{.075\textwidth} | >{\raggedleft\let\\\tabularnewline}p{.075\textwidth} | >{\raggedleft\let\\\tabularnewline}p{.075\textwidth} | >{\raggedleft\let\\\tabularnewline}p{.1\textwidth}} 
    \hline
     & $k$\TBstrut & $\mu_1$ & $\bar{x}_N$ & $\bar{x}_N^{\text{(t\&f)}}$ & $\bar{x}_N^{\text{(t\&f-ppr)}}$\\ 
    \hline
    full sample (no bias, A)\Tstrut & $200$ & $0$ & $-0.02$ & $-0.11$ & $-0.10$ \\ 
    & $200$\Bstrut & $0.3$ & $0.27$ & $0.27$ & $0.10$\\
    \hline
    significant studies (B)\Tstrut& $13$ & $0$ & $0.69$ & $0.46$ & $0.46$ \\
    & $38$\Bstrut & $0.3$ & $0.61$ & $0.45$ & $0.45$ \\
    \hline
    significant studies and $10\%$ of\Tstrut & $32$ & $0$ & $0.15$ & $0.11$ & $-0.002$ \\
    non-significant studies (C) & $55$\Bstrut & $0.3$ & $0.51$ & $0.42$ & $0.40$\\
 \hline
\end{tabular}
    \caption[The trim-and-fill method to correct effect size estimates.]{Results for the trim-and-fill method and the adapted trim-and-fill method to correct for effect size estimates. The correction was applied to the studies shown in Figure~\ref{fig:funnel_plot}. The global estimate for $V_{n_j}$ was calculated by using the arithmetic mean. The global estimate $\bar{x}_N$ was calculated according to Eq.~\ref{eq:global_estimator_precision_weight} but with theoretical parameter values replaced by empirical estimates.}
    \label{tab:trim_and_fill}
  \end{center}
\end{table}
If there is reason to belief that the $\ppr_j$ represents the publication probability of a finding $j$ below the significance threshold better than the decaying probability assumed by Duval and Tweedie, I propose the following adjustment to their trim-and-fill method, to be executed after the first execution of step~\ref{itm:transform} and before the first execution of step~\ref{itm:center} described above:
\begin{enumerate}[start=1,label={2.\arabic*.}]
    \item Set $i = 1$, but instead of starting out with $k_0^{(0)} = 0$, set it equal to the minimum of the number of all significant findings and the number of all non-significant findings reweighted by their respective publication probability $\ppr_j$, that is,
    $$k_0^{(0)} = \text{min}(\sum_{j=1}^{k} \delta(V_{n_j}),\sum_{j=1}^{k} (1-\delta(V_{n_j}))/\ppr_j) $$
    where $\delta(\cdot)$ is a decision function taking $S_n$ as input and returning $1$ if $V_n$ lies beyond the predefined significance threshold and $0$ otherwise.
    \item Trim off the $k_0^{(0)}$ must positive findings and calculate the global estimate $\hat{\theta}_{\text{tot}}^{(i)}$ based on the remaining data.
    \item Enter the trim-and-fill algorithm at step~\ref{itm:center} and iterate until convergence.
\end{enumerate}
Unless there is no publication present or the publication probability of non-significant studies is larger than assumed, the corrections outlined above improve the performance of the trim-and-fill method as can also be observed in Table~\ref{tab:trim_and_fill}.
%toadd: show that this adjustment helps improve the correction
%toadd: maybe add algo in which you simply trim off all significant finding / or number of significant findings corresponding to number of omitted studies and recalculate globa lmean

\subsection{Effect size correction based on \texorpdfstring{$p$}{p}-curves}
The $p$-curve described in Section~\ref{subsec:p-curve} cannot only be used for the detection of publication bias, but also for its correction. The correction method put forward by \citet{simonsohn_pcurve_correction_2014} leverages again on the different distributions of $p$-values depending on whether the null hypothesis is true or false, respectively, and whether $p$-hacking is present or absent, respectively (see Section~\ref{subsec:p-curve} for details). In short, the researcher calculates the $pp$-values for the published findings for a range of possible true effect sizes and then checks to resulting $p$-curves for correspondence to a uniform distribution. In detail, the procedure goes as follows (assuming that findings result from a one-sided superiority test):
\begin{enumerate}
    \item Let $s_{n_1},\dots,s_{n_l}$ be a set of published and significant test statistics with ${S_{n_j}\sim \mathcal{N}(\mu, \sigma^2/n_j)}$. Create a set of possible candidates $\{\mu_1,\dots,\mu_m\}$ for $\mu$.
    \item For each $\mu_i$ calculate the corresponding $p$-values for $s_{n_1},\dots,s_{n_l}$, that is, $p_{ij} = \Pr(S_{n_j} > s_{n_j}\mid \mu_i) = 1-F_{S_n}(\cdot \mid \mu_i)$, with $F_{S_{n_j}}(s_{n_j} \mid \mu_i)$ denoting the cumulative distribution function of $S_{n_j}$ given $\mu = \mu_i$.
    \item Transform the $p_{ij}$-values into $pp_{ij}$-values by conditioning on the fact that all observed $p$-values were significant and thus below the pre-defined Type I error rate $\alpha$:
    \begin{alignat*}{1}
    pp_{ij} = \Pr(P<p_{ij} \mid p_{ij} < 1-\beta_{ij}, \mu_i) = \frac{p_{ij}-\beta_{ij}}{1-\beta{ij}}.
    \end{alignat*}
    \item For each $\mu_i$, compare the corresponding $p$-curve consisting of $pp$-values $pp_{i1},\dots,pp_{il}$ to the uniform distribution. Simonsohn \textit{et al.} suggested using the Kolmogorov-Smirnov statistic (see for example \citet[p.~155]{barlow_statistics_1989}), as a measure of distance between of the empirical $p$-curve and the uniform distribution, that is,
    $$D_i = \max_j(|pp_{ij}-1/\alpha|).$$
    \item Pick the $\mu_i$ with the lowest $D_i$ as corrected estimate of the true effect size $\mu$.
\end{enumerate}
Table~\ref{tab:p_curve_correction} shows the results for the corrections applied to the studies shown in Figure~\ref{fig:funnel_plot}. As shown before (see Table~\ref{tab:p_curve}), the $p$-curve discards all information from non-significant studies and therefore is not able to distinguish between different extents of `pure' publication bias, that is, publication bias that only originates from a higher publication probability for significant results. 
\begin{table}[h!]
  \begin{center}
    \begin{tabular}{ >{\raggedright\let\\\tabularnewline}p{.4\textwidth} | >{\raggedleft\let\\\tabularnewline}p{.1\textwidth}| >{\raggedleft\let\\\tabularnewline}p{.1\textwidth} | >{\raggedleft\let\\\tabularnewline}p{.1\textwidth} | >{\raggedleft\let\\\tabularnewline}p{.1\textwidth}} 
        \hline
         & $k$\TBstrut & $\mu_1$ & $\bar{x}_N$ & $\bar{\mu}$ \\ 
        \hline
        full sample (no bias, A)\Tstrut & $200$ & $0$ & $-0.02$  & $0.41$ \\ 
        & $200$\Bstrut & $0.3$ & $0.27$  & $0.25$\\
        \hline
        significant studies (B)\Tstrut& $13$ & $0$ & $0.69$ & $0.41$ \\
        & $38$\Bstrut & $0.3$ & $0.61$ & $0.25$ \\
        \hline
        significant studies and $10\%$ of\Tstrut & $32$ & $0$ & $0.15$ & $0.41$ \\
        non-significant studies (C) & $55$\Bstrut & $0.3$ & $0.25$ & $0.27$ \\
     \hline
    \end{tabular}
    \caption[Using the $p$-curve to correct for publication bias.]{The results of the bias correction based on the $p$-curve method. The correction was applied to the studies shown in Figure~\ref{fig:funnel_plot}, the global estimate $\bar{x}_N$ was calculated according to Eq.~\ref{eq:global_estimator_precision_weight} with the theoretical parameter values replaced by empirical estimates.}
    \label{tab:p_curve_correction}
  \end{center}
\end{table}
Simonsohn \textit{et al.} showed that effect size corrections using the $p$-curve outperforms the standard trim-and-fill method in the case of publication bias. If $p$-hacking or HARKing is the cause of the publication bias (and not only bias based on preferential publishing of significant findings alone), then the $p$-curve correction still performs well, but underestimates the true effect size.

\subsection{Maximising the truncated likelihood}
Yet another possibility to correct biased effect size estimates is possible by exploiting the truncated likelihood function as outlined in Section~\ref{subsec:pub_prob and trunc_dist}, Eq.~\ref{eq:truncated_likelihood}. Let $v_{n_1},\dots,v_{n_k}$ be the variance stabilised and standard normally distributed test statistics for the true population mean $\mu$ in $k$ studies with $n_j$ data points each. Furthermore, let the publication probability of significant and non-significant studies be $1$ and $\pi$, respectively, with $\pi$ being constant. We can then write the truncated likelihood for the $\mu$ as:
\begin{align*}
        \mathcal{L}^{*}(\mu \mid v_{n_1},\dots,v_{n_k}) = \frac{\mathcal{L}(\mu \mid v_{n_1},\dots,v_{n_k})}{\E[\ppr(V_{n_j},\pi)\mid \mu]}\prod_{j=1}^k \ppr(V_{n_j},\pi).
\end{align*}
Estimated values or both $\mu$ and $\pi$ can now easily be computed by maximising the likelihood through grid-search optimisation: 
\begin{enumerate}
    \item Define a set of candidate values $\{\mu_1,\dots,\mu_m\}$ and $\{\pi_1,\dots,\pi_n\}$ for $\mu$ and $\pi$, respectively.
    \item For each combination of $\mu$ and $\pi$, calculate the likelihood.
    \item Choose the candidate value for $\mu$ and $\pi$ that yields the highest likelihood.
\end{enumerate}
As Table~\ref{tab:maximise_trunc_likelihood} shows, this approach yields corrected estimates for $\mu$ and $\pi$ that are rather close to the unbiased effect sizes and performs robustly across different scenarios.
\begin{table}[h!]
  \begin{center}
    \begin{tabular}{ >{\raggedright\let\\\tabularnewline}p{.4\textwidth} | >{\raggedleft\let\\\tabularnewline}p{.075\textwidth}| >{\raggedleft\let\\\tabularnewline}p{.075\textwidth} | >{\raggedleft\let\\\tabularnewline}p{.075\textwidth} | >{\raggedleft\let\\\tabularnewline}p{.075\textwidth} | >{\raggedleft\let\\\tabularnewline}p{.1\textwidth}} 
    \hline
     & $k$\TBstrut & $\mu_1$ & $\bar{x}_N$ & $\hat{\mu}$ & $\hat{\pi}$\\ 
    \hline
    full sample (no bias, A)\Tstrut & $200$ & $0$ & $-0.02$ & $-0.02$ & $0.66$ \\ 
    \Bstrut & $200$\Bstrut & $0.3$ & $0.27$ & $0.27$ & $1$\\
    \hline
    significant studies (B)\Tstrut & $13$ & $0$ & $0.69$ & $0.31$ & $0$ \\
    \Bstrut & $38$\Bstrut & $0.3$ & $0.61$ & $0.31$ & $0$ \\
    \hline
    significant studies and $10\%$ of\Tstrut & $32$ & $0$ & $0.15$ & $-0.07$ & $0.05$ \\
    non-significant studies (C)\Bstrut & $55$\Bstrut & $0.3$ & $0.51$ & $0.33$ & $0.16$\\
 \hline
\end{tabular}
    \caption[Maximising the truncated likelihood to correct for publication bias.]{Results for the effect size corrections and publication probability estimates based on the maximisation of the truncated likelihood. The correction was applied to the studies shown in Figure~\ref{fig:funnel_plot}. The global estimate $\bar{x}_N$ was calculated according to Eq.~\ref{eq:global_estimator_precision_weight} but with theoretical parameter values replaced by empirical estimates.}
    \label{tab:maximise_trunc_likelihood}
  \end{center}
\end{table}
\chapter{Conclusion}
\label{cha:conclusion_and_limitations}
\epigraph{\centering \textit{`In theory, practice is simple.'}}{--- Trygve M. H. Reenskaug}%\par Professor emeritus of informatics at the University of Oslo}
In this thesis, I presented an overview of a range of methods to construct robust and comparable measures for statistical evidence. In addition, I presented different approaches to detect and correct publication bias in effect size measures. Even though all of these methods and approaches are backed up by solid theoretical arguments, they often hinge on assumptions about the nature of scientific studies and the publication process. Assumptions which might be true but which might also be completely wrong.\par
For example, all tests introduced in this thesis as well as my adaptions thereof assume that individual studies are independent, that there is no between-study heterogeneity and that there is a fixed global effect. All these assumptions might be and are usually violated in real-life scientific practice.\par
Hence, it is important to keep in mind that none of the methods explained can be regarded as a `silver bullet' to fight publication bias---this is especially true, when one only relies on one of these methods alone. For example, it should have become clear that the reliance on the funnel plot as a diagnostic tool and the trim-and-fill approach as a corrective method is woefully inadequate to capture the wide range of forms in which publication bias can appear. Hence, it should become common practice to use more than one approach with different underlying assumptions to detect and correct publication bias.\par
In this spirit, I would like to pursue this topic with the following three goals in mind:
\begin{description}[leftmargin=!,labelwidth=\widthof{\bfseries $\boldsymbol{E_4}$}]
    \item [Systematic evaluation of performance] The scope of this did not allow me to systematically test the performance of all methods described. Hence, I would like to do this for a range of different scenarios of publication bias based on different assumptions and empirical observations. This is important since many of the published methods only perform well in very specific scenarios and completely fail if certain theoretical assumptions are not met. This also holds for my own methods presented in this thesis. It is therefore crucial to systematically asses the performance of all of these methods in order to construct reliable usage recommendations for statisticians and non-statisticians alike.\\
    \item[Extend assessment to additional methods] This thesis is not an exhaustive summary of methods pertaining to the detection and correction of publication bias. For example, I did not go into methods accounting for between-study heterogeneity and applying random effects models \citep{piao_copaslike_2018} or Bayesian approaches \citep{cleary_application_1997, andrews_identification_2017}, and hardly touched maximum likelihood-based approach such as the ones proposed by \citet{copas_what_1999}. Since these approaches often hinge on assumptions that differ from those underlying the methods described in this thesis, it is most likely advantageous to include them at a later step in order to increase the range of scenarios in which publication bias can be detected.\\
    \item [Creating ensemble models] My simulations have shown that many standard methods to detect and correct publication bias fail in certain situations and can even be further improved from a practical and sometimes even from a theoretical point of view. Given the fact that different methods hinge on different assumptions and thus show different performances depending on the concrete scenario, it should be clear that reliance on only one approach to detect or correct publication bias is prone to yield misleading results. It is therefore desirable to combine some of these methods into ensemble models for the detection and correction of publication bias. Since many of the methods described in this thesis are based on similar or related statistical measure, aggregating results across different methods should be possible.
\end{description}
However, I must point out the observation that the best statistical tools to correct biased estimates pale in comparison with non-statistical methods when it comes to efficacy to prevent bias in scientific results. For example, a strict distinction between exploratory and confirmatory studies---with mandatory pre-registration for the latter---could alleviate a lot of the problems outlined in this thesis. For a start, it would be possible to select only those studies that were intended to be confirmatory for meta-analysis and ignore the rest. In addition, it would be possible to calculate the publication probability of non-significant findings to correct for any kind of publication bias. And last but definitely not least it would help increasing the overall quality of scientific work if researchers are forced to think about study designs and statistical methods before the onset of the study---hopefully also by consulting a statistician before and not, as is often the case, after a study. Because, as \citet[p.~17]{fisher_presidential_1938} already remarked: `To consult a statistician after an experiment is finished is often merely to ask him to conduct a \textit{post mortem} examination. He can perhaps say what the experiment died of."
\vspace*{\fill}

%Assumptions that might and often are violated in real life:
%\begin{itemize}
    %\item Sample Variance is known
    %\item Sample is large enough for the central limit theorem to hold
    %\item Access to sample size, mean and sample standard deviation is given
%\end{itemize}

%Also, all my simulations of publication bias were based on hypothesis test using the variance stabilised transformation $V_n$ as described in Eq.~\ref{eq:Vn_stud}. However, in many cases researcher use the $z$-test instead or other inappropriate tests which increases the error rates
% .There also exist several techniques to model selection and correct for publication bias using weighted distribution theory (Hedges 1984; Iyengar and Greenhouse 1988; Hedges 1992; Dear and Begg 1992), Bayesian statistics (Cleary and Casella 1997), and maximum likelihood (Copas 1999). In addition to selection models, sensitivity analyses are often used to ascertain the severity of the bias (Rosenthal 1979; Gleser and Olkin 1996; Duval and Tweedie 2000) Stanley and Jarrell (1989) meta-regression analysis
%\input{example-short-chapter}   %% remove this line to get rid of the example chapter
%\input{example-style-chapter}   %% remove this line to get rid of the style chapter

%% include tex file chapters:
% \include{introduction}        %% this is a suggestion: you have to create this file on demand
% \include{problem}             %% this is a suggestion: you have to create this file on demand
% \include{solution}            %% this is a suggestion: you have to create this file on demand
% \include{evaluation}          %% this is a suggestion: you have to create this file on demand
% \include{outlook}             %% this is a suggestion: you have to create this file on demand

\input{template/declaration_EPFL}  %% Statutory Declaration

\appendix                       %% closes main document, appendix follows until end; only available in book-classes
\addpart*{Appendix}             %% adding Appendix to tableofcontents
\chapter{Package Overview and Code Repository}
\label{cha:appendix}
\section{Repository on Github}
All simulations and figures presented in this thesis were produced using the \textsf{R} programming language. The corresponding code can be found on Github under \href{https://github.com/segrue/2019_WeightOfStatisticalEvidence.git}{https://github.com/segrue/2019\_WeightOfStatisticalEvidence.git}. For remarks, corrections or inquiries about the thesis you can reach me under the contact information given on my website \href{www.servangrueninger.ch}{www.servangrueninger.ch}.\newpage

\section{\textsf{R}-packages}
\begin{lstlisting}[
    basicstyle=\tiny, %or \small or \footnotesize etc.
][language=R]
R version 3.6.0 (2019-04-26)
Platform: x86_64-pc-linux-gnu (64-bit)
Running under: Ubuntu 16.04.6 LTS

Matrix products: default
BLAS:   /usr/lib/libblas/libblas.so.3.6.0
LAPACK: /usr/lib/lapack/liblapack.so.3.6.0

Random number generation:
 RNG:     Mersenne-Twister 
 Normal:  Inversion 
 Sample:  Rounding 
 
attached base packages:
[1] grid stats graphics grDevices utils datasets methods base     

other attached packages:
 [1] reshape2_1.4.3      RColorBrewer_1.1-2  poibin_1.3          NMOF_1.6-0         
 [5] metasens_0.3-2      meta_4.9-5          gridExtra_2.3       ggpubr_0.2         
 [9] magrittr_1.5        fitdistrplus_1.0-14 npsurv_0.4-0        lsei_1.2-0         
[13] survival_2.44-1.1   MASS_7.3-51.4       extrafont_0.17      data.table_1.12.2  
[17] cowplot_0.9.4       ggplot2_3.1.1      

loaded via a namespace (and not attached):
 [1] Rcpp_1.0.1       pillar_1.4.0     compiler_3.6.0   plyr_1.8.4       tools_3.6.0     
 [6] nlme_3.1-140     tibble_2.1.1     gtable_0.3.0     lattice_0.20-38  pkgconfig_2.0.2 
[11] rlang_0.3.4      Matrix_1.2-17    rstudioapi_0.10  parallel_3.6.0   Rttf2pt1_1.3.7  
[16] stringr_1.4.0    metafor_2.1-0    withr_2.1.2      dplyr_0.8.1      tidyselect_0.2.5
[21] glue_1.3.1       R6_2.4.0         purrr_0.3.2      extrafontdb_1.0  scales_1.0.0    
[26] splines_3.6.0    assertthat_0.2.1 colorspace_1.4-1 stringi_1.4.3    lazyeval_0.2.2  
[31] munsell_0.5.0    crayon_1.3.4

\end{lstlisting}






\printbibliography              %% remove, if using BibTeX instead of biblatex
% \include{further_ressources}  %% this is a suggestion: you have to create this file on demand


%%%% end{document}
\end{document}
%% vim:foldmethod=expr
%% vim:fde=getline(v\:lnum)=~'^%%%%\ .\\+'?'>1'\:'='
%%% Local Variables:
%%% mode: latex
%%% mode: auto-fill
%%% mode: flyspell
%%% eval: (ispell-change-dictionary "en_US")
%%% TeX-master: "main"
%%% End:
